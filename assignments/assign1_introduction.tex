% Preamble for problem sets, exams, and project descriptions
% Applied Quantitative Methods for the Social Sciences II

\documentclass[12pt, a4paper]{article}
\usepackage[margin = 2cm]{geometry}
\usepackage{graphicx}
\usepackage[english]{babel}
\usepackage[utf8]{inputenc}
\usepackage[colorlinks = TRUE, linkcolor = blue, urlcolor = blue, citecolor = blue]{hyperref}
\usepackage{setspace}
\setstretch{1.25}
\renewcommand*\rmdefault{ppl}

\usepackage[]{titlesec}
    \titleformat*{\section}{\large\bf}
    \titleformat*{\subsection}{\normalsize\bf}
    \titleformat*{\subsubsection}{\normalsize\it}

% For code blocks
\usepackage{listings}
\usepackage{xcolor}
\lstset{
    basicstyle=\ttfamily\small,
    backgroundcolor=\color{gray!10},
    frame=single,
    framerule=0pt,
    breaklines=true,
    columns=fullflexible,
    keepspaces=true
}

% For inline code
\newcommand{\code}[1]{\texttt{#1}}

% Enumerate with letters
\usepackage{enumitem}

\textbf{\large Assignment 1: Setting Up Workflow}\\\vspace{10pt}
\end{center}

\vspace{10pt}
\noindent
\textbf{\large Instructions:}

\vspace{10pt}
\begin{itemize}
\setlength\itemsep{0pt}
  \item {\color{red}{\textbf{Deadline}}}: \textbf{February 12, before class}
  \item This assignment walks you through setting up Git and GitHub
  \item You will use this setup to submit all assignments throughout the course
  \item Complete all the tasks below and send me the link to your GitHub repository
\end{itemize}

\vspace{20pt}
\tableofcontents

\section{Task 1: Create a GitHub Account}

If you don't already have a GitHub account, create one at \url{https://github.com}.

\begin{enumerate}
  \item Go to \url{https://github.com} and click ``Sign up''
  \item Choose a \textbf{professional username}---you may use this for years in your academic career
  \item Use your university email or a professional email address
  \item Complete the verification process
  % \item Optionally: Add a profile picture and brief bio
\end{enumerate}

\noindent\rule{\textwidth}{1pt}
\section{Task 2: Create a Repository for This Course}

Create a new \textbf{public} repository. Name it something like \code{aqmss2} or \code{quant-methods-2026}.

\subsection{Option A: Using the GitHub Web Interface}

\begin{enumerate}
  \item Click the ``+'' icon in the top-right corner of GitHub
  \item Select ``New repository''
  \item Fill in the form:
    \begin{itemize}
      \item \textbf{Repository name}: \code{aqmss2} (or similar)
      \item \textbf{Description}: ``Assignments for AQMSS II, Spring 2026''
      \item \textbf{Visibility}: Select \textbf{Public}
      \item Check the box ``Add a README file''
    \end{itemize}
  \item Click ``Create repository''
\end{enumerate}

\subsection{Option B: Using the Command Line}

First, make sure Git is installed on your computer. Then run:

\begin{lstlisting}
# Create a new directory and initialize Git
mkdir aqmss2
cd aqmss2
git init

# Create a README file
echo "# AQMSS II - Problem Sets" > README.md

# Stage and commit the file
git add README.md
git commit -m "Initial commit"

# Set up the remote repository (create it on GitHub first, without README)
git branch -M main
git remote add origin https://github.com/YOUR-USERNAME/aqmss2.git
git push -u origin main
\end{lstlisting}

\textbf{Note}: Replace \code{YOUR-USERNAME} with your actual GitHub username.

\subsection{Option C: Using RStudio}

\begin{enumerate}
  \item First, create an empty repository on GitHub (without README)
  \item In RStudio: File $\rightarrow$ New Project $\rightarrow$ Version Control $\rightarrow$ Git
  \item Paste your repository URL: \code{https://github.com/YOUR-USERNAME/aqmss2.git}
  \item Choose a location on your computer
  \item Click ``Create Project''
\end{enumerate}

\noindent\rule{\textwidth}{1pt}
\section{Task 3: Edit the README File}

Your README is the ``front page'' of your repository. Edit it to include:

\begin{itemize}
  \item Your name
  \item A brief description (e.g., ``Assignments for AQMSS II, Spring 2026'')
  \item Optionally, a list of what will be in the repository
\end{itemize}

\subsection{Option A: On the Web}

\begin{enumerate}
  \item Click on \code{README.md} in your repository
  \item Click the pencil icon (edit) in the top-right of the file view
  \item Make your changes using Markdown syntax
  \item Scroll down and click ``Commit changes''
  \item Add a commit message like ``Update README with my info''
\end{enumerate}

\subsection{Option B: Command Line}

\begin{lstlisting}
# Edit README.md with any text editor, then:
git add README.md
git commit -m "Update README with my info"
git push
\end{lstlisting}

\subsection{Option C: RStudio}

\begin{enumerate}
  \item Edit the \code{README.md} file in the Files pane
  \item Go to the Git pane (usually top-right)
  \item Check the box next to \code{README.md} to stage it
  \item Click ``Commit''
  \item Write a commit message and click ``Commit''
  \item Click ``Push'' to upload to GitHub
\end{enumerate}

\noindent\rule{\textwidth}{1pt}
\section{Task 4: Create a Folder and R File}

Create a folder called \code{assignment1} and add your first R file.

\subsection{Option A: On the Web}

\begin{enumerate}
  \item Click ``Add file'' $\rightarrow$ ``Create new file''
  \item In the filename box, type: \code{assignment1/assign1.R}
    \begin{itemize}
      \item This creates the folder and file at once
    \end{itemize}
  \item Add the following content:
\begin{lstlisting}
# Assignment 1
# AQMSS II, Spring 2026
# [Your Name]

# This file will contain my solutions for Assignment 1
print("Hello, Git!")
\end{lstlisting}
  \item Scroll down and commit with message ``Add assign1.R''
\end{enumerate}

\subsection{Option B: Command Line}

\begin{lstlisting}
# Create the folder
mkdir assignment1

# Create the R file (use any text editor)
# Then stage, commit, and push:
git add assignment1/assign1.R
git commit -m "Add assign1.R"
git push
\end{lstlisting}

\subsection{Option C: RStudio}

\begin{enumerate}
  \item Create a new folder \code{assignment1} in the Files pane
  \item Create a new R script: File $\rightarrow$ New File $\rightarrow$ R Script
  \item Add the header content and save as \code{assignment1/assign1.R}
  \item In the Git pane, stage the new file, commit, and push
\end{enumerate}

\noindent\rule{\textwidth}{1pt}
\section{Task 5: Load a Dataset and Make a Plot}

In this task, you will download a public dataset, add it to your repository, load it in R, and create a simple plot.

\subsection{Download the data}

We will use the Gapminder dataset, which contains country-level data on life expectancy, GDP per capita, and population from 1952 to 2007.

\begin{enumerate}
  \item Create a \code{data/} folder in your repository
  \item In R, run the following to download the data and save it as a CSV file:
\begin{lstlisting}
install.packages("gapminder")
library(gapminder)
write.csv(gapminder, "data/gapminder.csv", row.names = FALSE)
\end{lstlisting}
  \item Verify that the file \code{data/gapminder.csv} exists in your repository folder
\end{enumerate}

\subsection{Load the data and make a plot}

Create a new R script called \code{assignment1/ass1\_plot.R} with the following steps:

\begin{enumerate}
  \item Load the data from the CSV file:
\begin{lstlisting}
library(ggplot2)
gapminder <- read.csv("data/gapminder.csv")
\end{lstlisting}
  \item Explore the data briefly:
\begin{lstlisting}
head(gapminder)
str(gapminder)
\end{lstlisting}
  \item Create a plot showing how life expectancy has changed over time for a few countries of your choice. For example:
\begin{lstlisting}
countries <- c("Spain", "France", "Germany", "United Kingdom")
df <- gapminder[gapminder$country %in% countries, ]

ggplot(df, aes(x = year, y = lifeExp, color = country)) +
  geom_line() +
  geom_point() +
  labs(x = "Year", y = "Life expectancy",
       title = "Life expectancy over time") +
  theme_minimal()

ggsave("assignment1/ass1_plot.png", width = 7, height = 5)
\end{lstlisting}
  \item You can choose different countries or a different variable (e.g., \code{gdpPercap} or \code{pop})
\end{enumerate}

\subsection{Commit everything}

Stage and commit the data file, the R script, and the plot:

\begin{lstlisting}
git add data/gapminder.csv assignment1/ass1_plot.R assignment1/ass1_plot.png
git commit -m "Add gapminder data and first plot"
git push
\end{lstlisting}

\noindent\rule{\textwidth}{1pt}
\section{Task 6: View Your Commit History}

Check that your commits were recorded properly.

\subsection{Option A: On the Web}

\begin{enumerate}
  \item Go to your repository page on GitHub
  \item Click on ``Commits'' (or the clock icon with a number)
  \item You should see your commits listed with messages and timestamps
\end{enumerate}

\subsection{Option B: Command Line}

\begin{lstlisting}
git log --oneline
\end{lstlisting}

This shows a compact list of your commits.

\subsection{Option C: RStudio}

\begin{enumerate}
  \item In the Git pane, click ``History'' (clock icon)
  \item Browse through your commits
\end{enumerate}

\vspace{10pt}

\textbf{Take a screenshot} of your commit history showing at least 3--4 commits.

\noindent\rule{\textwidth}{1pt}
\section{SUBMISSION}

\begin{enumerate}
\item Send me an email with the URL of your GitHub repository\\
        (e.g., \code{https://github.com/username/aqmss2})
\item Requirements:

\end{enumerate}

\begin{itemize}
  \item Your repository is \textbf{public}
  \item It contains a README with your name
  \item It has a \code{data/} folder with the Gapminder CSV file
  \item It has a \code{assignment1/} folder with your R files and a saved plot
\end{itemize}

\clearpage
\section{Extra: Installing Git Locally}

If you want to use Git from the command line, you need to install it first.

\subsection{Installation}

\begin{itemize}
  \item \textbf{Mac}: Git comes pre-installed. Or install via Homebrew: \code{brew install git}
  \item \textbf{Windows}: Download from \url{https://git-scm.com/download/win}
  \item \textbf{Linux}: Use your package manager, e.g., \code{sudo apt install git}
\end{itemize}

\subsection{First-Time Configuration}

After installing, configure your identity (run once):

\begin{lstlisting}
git config --global user.name "Your Name"
git config --global user.email "your@email.com"
\end{lstlisting}

\subsection{Cloning an Existing Repository}

If you created the repository on GitHub first, download it to your computer:

\begin{lstlisting}
git clone https://github.com/YOUR-USERNAME/aqmss2.git
cd aqmss2
\end{lstlisting}

\subsection{Configuring RStudio}

To use Git in RStudio:
\begin{enumerate}
  \item Go to Tools $\rightarrow$ Global Options $\rightarrow$ Git/SVN
  \item Make sure ``Git executable'' points to your Git installation
  \item Restart RStudio
\end{enumerate}

\vspace{15pt}

\subsection*{Quick Reference on Common Git Commands}

\begin{tabular}{ll}
\textbf{Command} & \textbf{What it does} \\
\hline
\code{git status} & Show which files have changed \\
\code{git add <file>} & Stage a file for commit \\
\code{git add .} & Stage all changed files \\
\code{git commit -m "msg"} & Save staged changes with a message \\
\code{git push} & Upload commits to GitHub \\
\code{git pull} & Download changes from GitHub \\
\code{git log --oneline} & Show commit history (compact) \\
\code{git diff} & Show changes not yet staged \\
\end{tabular}

\clearpage
\section{Extra: Setting Up the Command Line on Windows}

If you are using Windows and want to follow the command line instructions in this assignment, you need to set up a Unix-like terminal first. The default Windows terminals (Command Prompt and PowerShell) use different commands and syntax from what is shown in this assignment and in most online tutorials.

\subsection{Recommended: Use Git Bash}

When you install Git for Windows (from \url{https://git-scm.com/download/win}), it includes \textbf{Git Bash}---a terminal emulator that provides a Unix-like command line environment on Windows. This is the easiest way to get started.

\begin{enumerate}
  \item Download and run the Git for Windows installer
  \item During installation, accept the default options. In particular:
    \begin{itemize}
      \item Select ``Use Git from Git Bash only'' (or the default option)
      \item Select ``Use bundled OpenSSH''
      \item Select ``Checkout Windows-style, commit Unix-style line endings''
    \end{itemize}
  \item After installation, you can open Git Bash by:
    \begin{itemize}
      \item Searching for ``Git Bash'' in the Start menu
      \item Right-clicking in any folder and selecting ``Git Bash Here''
    \end{itemize}
\end{enumerate}

Git Bash supports all the commands used in this assignment (\code{mkdir}, \code{cd}, \code{echo}, \code{git}, etc.) with the same syntax as on Mac and Linux.

\subsection{Alternative: Windows Terminal with PowerShell}

If you prefer to use PowerShell, be aware of these differences:

\begin{itemize}
  \item Most Git commands work the same (\code{git add}, \code{git commit}, etc.)
  \item File paths use backslashes (\code{\textbackslash}) instead of forward slashes (\code{/}), though Git accepts both
  \item The \code{echo} command syntax differs: use \code{Set-Content} or \code{Out-File} instead of \code{>} redirection
  \item Some Unix commands (\code{touch}, \code{cat}) are not available by default
\end{itemize}

\subsection{Setting up your PATH}

After installing Git, make sure it is accessible from your terminal:

\begin{lstlisting}
# In Git Bash or PowerShell, run:
git --version
\end{lstlisting}

If this returns a version number (e.g., \code{git version 2.43.0}), you are ready to go. If not, you may need to add Git to your system PATH:

\begin{enumerate}
  \item Open Settings $\rightarrow$ System $\rightarrow$ About $\rightarrow$ Advanced system settings
  \item Click ``Environment Variables''
  \item Under ``System variables'', find \code{Path} and click ``Edit''
  \item Add the path to your Git installation (usually \code{C:\textbackslash Program Files\textbackslash Git\textbackslash cmd})
  \item Click OK and restart your terminal
\end{enumerate}

\clearpage
\section{Extra: Using Positron Instead of RStudio}

In this course, we recommend using \textbf{Positron} (\url{https://positron.posit.co}) as your IDE instead of RStudio. Positron is a next-generation data science IDE built by Posit (the same company behind RStudio). It is based on VS Code and designed for R and Python. Here is what differs from RStudio when completing this assignment.

\subsection{Installing Positron}

\begin{enumerate}
  \item Download Positron from \url{https://positron.posit.co}
  \item Install it like any other application
  \item On first launch, Positron will detect your R installation automatically
\end{enumerate}

\subsection{Differences for This Assignment}

The following table maps the RStudio instructions in this assignment to their Positron equivalents:

\vspace{10pt}

\begin{tabular}{p{0.45\textwidth} p{0.45\textwidth}}
\textbf{RStudio} & \textbf{Positron} \\
\hline
File $\rightarrow$ New Project $\rightarrow$ Version Control $\rightarrow$ Git & File $\rightarrow$ New Folder, then open the folder. Use the built-in terminal (\code{Ctrl+`}) to run \code{git clone}. \\[8pt]
Git pane (top-right) & Source Control panel in the left sidebar (branch icon), or \code{Ctrl+Shift+G} \\[8pt]
Check box next to file to stage it & Click the \code{+} icon next to the file in the Source Control panel \\[8pt]
Click ``Commit'' button & Click the checkmark icon in the Source Control panel, type your message in the text box \\[8pt]
Click ``Push'' button & Click the \code{...} menu in Source Control $\rightarrow$ Push, or use the terminal: \code{git push} \\[8pt]
Click ``History'' (clock icon) & Use the terminal: \code{git log --oneline}, or install the ``Git Graph'' extension for a visual history \\[8pt]
Files pane (bottom-right) & Explorer panel in the left sidebar (top icon), or \code{Ctrl+Shift+E} \\[8pt]
File $\rightarrow$ New File $\rightarrow$ R Script & File $\rightarrow$ New File, then select ``R File'' \\[8pt]
Tools $\rightarrow$ Global Options $\rightarrow$ Git/SVN & Git is detected automatically. If not, open Settings (\code{Ctrl+,}) and search for ``git.path'' \\
\end{tabular}

\subsection{Key Advantages of Positron}

\begin{itemize}
  \item \textbf{Integrated terminal}: Positron has a built-in terminal (\code{Ctrl+`}) where you can run Git commands directly---no need to switch to a separate application
  \item \textbf{Better Git integration}: The Source Control panel shows diffs, staged changes, and commit history in one place
  \item \textbf{Extensions}: You can install extensions (e.g., ``Git Graph'' for visual commit history, ``R'' for enhanced R support) from the Extensions panel
  \item \textbf{Multiple languages}: Positron works equally well with R and Python, which is useful if you work across both
\end{itemize}

\textbf{Note}: All the command line instructions in this assignment work identically regardless of whether you use RStudio, Positron, or a standalone terminal. The differences only apply when using the graphical interface.

\clearpage
\section{Extra: Using Sublime Text 4 as a Code Editor}

Sublime Text is a fast, lightweight code editor. Unlike RStudio or Positron, it is not an IDE---it does not come with an R console, file browser, or plot viewer out of the box. Instead, you write R code in Sublime Text and send it to a separate R console running alongside it. This minimal setup requires only two packages.

\subsection{Step 1: Install Sublime Text 4}

Download from \url{https://www.sublimetext.com} and install it.

\subsection{Step 2: Install Package Control}

Package Control is Sublime Text's package manager. To install it:

\begin{enumerate}
  \item Open the Command Palette: \code{Ctrl+Shift+P} (Windows/Linux) or \code{Cmd+Shift+P} (Mac)
  \item Type ``Install Package Control'' and press Enter
  \item Wait for the confirmation message
\end{enumerate}

\subsection{Step 3: Install R-IDE and SendCode}

Using Package Control, install two packages:

\begin{enumerate}
  \item Open the Command Palette (\code{Ctrl+Shift+P} / \code{Cmd+Shift+P})
  \item Type ``Package Control: Install Package'' and press Enter
  \item Search for \textbf{R-IDE} and install it (provides R syntax highlighting, code completions, and function signatures)
  \item Repeat the process and install \textbf{SendCode} (sends code from the editor to an external R console)
\end{enumerate}

\subsection{Step 4: Configure SendCode}

SendCode needs to know where to send your code. Open its settings:

\begin{enumerate}
  \item Go to Preferences $\rightarrow$ Package Settings $\rightarrow$ SendCode $\rightarrow$ Settings
  \item In the right-hand pane (user settings), paste the following:
\end{enumerate}

\begin{lstlisting}
{
    "prog": "Terminal"
}
\end{lstlisting}

On \textbf{Mac}, set \code{"prog"} to \code{"Terminal"} to send code to the built-in Terminal (where you will run R), or to \code{"iTerm"} if you use iTerm2. On \textbf{Windows}, set it to \code{"Cmder"}, \code{"ConEmu"}, or another terminal emulator. On \textbf{Linux}, set it to \code{"tmux"} or \code{"linux-terminal"}.

\subsection{Step 5: The Workflow}

\begin{enumerate}
  \item Open a terminal window and start R by typing \code{R} and pressing Enter
  \item In Sublime Text, open your \code{.R} file
  \item Place your cursor on a line of code and press \code{Ctrl+Enter} (Windows/Linux) or \code{Cmd+Enter} (Mac)---SendCode sends that line to the R console and moves to the next line
  \item To send a selection, highlight multiple lines and press the same shortcut
\end{enumerate}

That is the entire setup. You edit in Sublime Text, execute in the R console, and switch between them as needed. For Git operations, use the terminal.

\end{document}
