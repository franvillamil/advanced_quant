% Preamble for problem sets, exams, and project descriptions
% Applied Quantitative Methods for the Social Sciences II

\documentclass[12pt, a4paper]{article}
\usepackage[margin = 2cm]{geometry}
\usepackage{graphicx}
\usepackage[english]{babel}
\usepackage[utf8]{inputenc}
\usepackage[colorlinks = TRUE, linkcolor = blue, urlcolor = blue, citecolor = blue]{hyperref}
\usepackage{setspace}
\setstretch{1.25}
\renewcommand*\rmdefault{ppl}

\usepackage[]{titlesec}
    \titleformat*{\section}{\large\bf}
    \titleformat*{\subsection}{\normalsize\bf}
    \titleformat*{\subsubsection}{\normalsize\it}

% For code blocks
\usepackage{listings}
\usepackage{xcolor}
\lstset{
    basicstyle=\ttfamily\small,
    backgroundcolor=\color{gray!10},
    frame=single,
    framerule=0pt,
    breaklines=true,
    columns=fullflexible,
    keepspaces=true
}

% For inline code
\newcommand{\code}[1]{\texttt{#1}}

% Enumerate with letters
\usepackage{enumitem}

\textbf{\large Assignment 1: Setting Up Workflow}\\\vspace{10pt}
\end{center}

\vspace{10pt}
\noindent
\textbf{\large Instructions:}

\vspace{10pt}
\begin{itemize}
\setlength\itemsep{0pt}
  \item {\color{red}{\textbf{Deadline}}}: \textbf{February 12, before class}
  \item This assignment walks you through setting up Git and GitHub
  \item You will use this setup to submit all assignments throughout the course
  \item Complete all the tasks below and send me the link to your GitHub repository
\end{itemize}

\vspace{20pt}
\tableofcontents

\clearpage
\section{Task 1: Create a GitHub Account}

If you don't already have a GitHub account, create one at \url{https://github.com}.

\begin{enumerate}
  \item Go to \url{https://github.com} and click ``Sign up''
  \item Choose a \textbf{professional username}---you may use this for years in your academic career
  \item Use your university email or a professional email address
  \item Complete the verification process
  % \item Optionally: Add a profile picture and brief bio
\end{enumerate}

\noindent\rule{\textwidth}{1pt}
\section{Task 2: Create a Repository for This Course}

Create a new \textbf{public} repository. Name it something like \code{aqmss2} or \code{quant-methods-2026}.

\subsection{Option A: Using the GitHub Web Interface}

\begin{enumerate}
  \item Click the ``+'' icon in the top-right corner of GitHub
  \item Select ``New repository''
  \item Fill in the form:
    \begin{itemize}
      \item \textbf{Repository name}: \code{aqmss2} (or similar)
      \item \textbf{Description}: ``Assignments for AQMSS II, Spring 2026''
      \item \textbf{Visibility}: Select \textbf{Public}
      \item Check the box ``Add a README file''
    \end{itemize}
  \item Click ``Create repository''
\end{enumerate}

\subsection{Option B: Using the Command Line}

First, make sure Git is installed on your computer. Then run:

\begin{lstlisting}
# Create a new directory and initialize Git
mkdir aqmss2
cd aqmss2
git init

# Create a README file
echo "# AQMSS II - Problem Sets" > README.md

# Stage and commit the file
git add README.md
git commit -m "Initial commit"

# Set up the remote repository (create it on GitHub first, without README)
git branch -M main
git remote add origin https://github.com/YOUR-USERNAME/aqmss2.git
git push -u origin main
\end{lstlisting}

\textbf{Note}: Replace \code{YOUR-USERNAME} with your actual GitHub username.

\subsection{Option C: Using RStudio}

\begin{enumerate}
  \item First, create an empty repository on GitHub (without README)
  \item In RStudio: File $\rightarrow$ New Project $\rightarrow$ Version Control $\rightarrow$ Git
  \item Paste your repository URL: \code{https://github.com/YOUR-USERNAME/aqmss2.git}
  \item Choose a location on your computer
  \item Click ``Create Project''
\end{enumerate}

\noindent\rule{\textwidth}{1pt}
\section{Task 3: Edit the README File}

Your README is the ``front page'' of your repository. Edit it to include:

\begin{itemize}
  \item Your name
  \item A brief description (e.g., ``Assignments for AQMSS II, Spring 2026'')
  \item Optionally, a list of what will be in the repository
\end{itemize}

\subsection{Option A: On the Web}

\begin{enumerate}
  \item Click on \code{README.md} in your repository
  \item Click the pencil icon (edit) in the top-right of the file view
  \item Make your changes using Markdown syntax
  \item Scroll down and click ``Commit changes''
  \item Add a commit message like ``Update README with my info''
\end{enumerate}

\subsection{Option B: Command Line}

\begin{lstlisting}
# Edit README.md with any text editor, then:
git add README.md
git commit -m "Update README with my info"
git push
\end{lstlisting}

\subsection{Option C: RStudio}

\begin{enumerate}
  \item Edit the \code{README.md} file in the Files pane
  \item Go to the Git pane (usually top-right)
  \item Check the box next to \code{README.md} to stage it
  \item Click ``Commit''
  \item Write a commit message and click ``Commit''
  \item Click ``Push'' to upload to GitHub
\end{enumerate}

\noindent\rule{\textwidth}{1pt}
\section{Task 4: Create a Folder and R File}

Create a folder called \code{assignment1} and add your first R file.

\subsection{Option A: On the Web}

\begin{enumerate}
  \item Click ``Add file'' $\rightarrow$ ``Create new file''
  \item In the filename box, type: \code{assignment1/assign1.R}
    \begin{itemize}
      \item This creates the folder and file at once
    \end{itemize}
  \item Add the following content:
\begin{lstlisting}
# Assignment 1
# AQMSS II, Spring 2026
# [Your Name]

# This file will contain my solutions for Assignment 1
print("Hello, Git!")
\end{lstlisting}
  \item Scroll down and commit with message ``Add assign1.R''
\end{enumerate}

\subsection{Option B: Command Line}

\begin{lstlisting}
# Create the folder
mkdir assignment1

# Create the R file (use any text editor)
# Then stage, commit, and push:
git add assignment1/assign1.R
git commit -m "Add assign1.R"
git push
\end{lstlisting}

\subsection{Option C: RStudio}

\begin{enumerate}
  \item Create a new folder \code{assignment1} in the Files pane
  \item Create a new R script: File $\rightarrow$ New File $\rightarrow$ R Script
  \item Add the header content and save as \code{assignment1/assign1.R}
  \item In the Git pane, stage the new file, commit, and push
\end{enumerate}

\noindent\rule{\textwidth}{1pt}
\section{Task 5: Load a Dataset and Make a Plot}

In this task, you will download a public dataset, add it to your repository, load it in R, and create a simple plot.

\subsection{Download the data}

We will use the Gapminder dataset, which contains country-level data on life expectancy, GDP per capita, and population from 1952 to 2007.

\begin{enumerate}
  \item Create a \code{data/} folder in your repository
  \item In R, run the following to download the data and save it as a CSV file:
\begin{lstlisting}
install.packages("gapminder")
library(gapminder)
write.csv(gapminder, "data/gapminder.csv", row.names = FALSE)
\end{lstlisting}
  \item Verify that the file \code{data/gapminder.csv} exists in your repository folder
\end{enumerate}

\subsection{Load the data and make a plot}

Create a new R script called \code{assignment1/ass1\_plot.R} with the following steps:

\begin{enumerate}
  \item Load the data from the CSV file:
\begin{lstlisting}
library(ggplot2)
gapminder <- read.csv("data/gapminder.csv")
\end{lstlisting}
  \item Explore the data briefly:
\begin{lstlisting}
head(gapminder)
str(gapminder)
\end{lstlisting}
  \item Create a plot showing how life expectancy has changed over time for a few countries of your choice. For example:
\begin{lstlisting}
countries <- c("Spain", "France", "Germany", "United Kingdom")
df <- gapminder[gapminder$country %in% countries, ]

ggplot(df, aes(x = year, y = lifeExp, color = country)) +
  geom_line() +
  geom_point() +
  labs(x = "Year", y = "Life expectancy",
       title = "Life expectancy over time") +
  theme_minimal()

ggsave("assignment1/ass1_plot.png", width = 7, height = 5)
\end{lstlisting}
  \item You can choose different countries or a different variable (e.g., \code{gdpPercap} or \code{pop})
\end{enumerate}

\subsection{Commit everything}

Stage and commit the data file, the R script, and the plot:

\begin{lstlisting}
git add data/gapminder.csv assignment1/ass1_plot.R assignment1/ass1_plot.png
git commit -m "Add gapminder data and first plot"
git push
\end{lstlisting}

\noindent\rule{\textwidth}{1pt}
\section{Task 6: View Your Commit History}

Check that your commits were recorded properly.

\subsection{Option A: On the Web}

\begin{enumerate}
  \item Go to your repository page on GitHub
  \item Click on ``Commits'' (or the clock icon with a number)
  \item You should see your commits listed with messages and timestamps
\end{enumerate}

\subsection{Option B: Command Line}

\begin{lstlisting}
git log --oneline
\end{lstlisting}

This shows a compact list of your commits.

\subsection{Option C: RStudio}

\begin{enumerate}
  \item In the Git pane, click ``History'' (clock icon)
  \item Browse through your commits
\end{enumerate}

\vspace{10pt}

\textbf{Take a screenshot} of your commit history showing at least 3--4 commits.

\noindent\rule{\textwidth}{1pt}
\section{Submission}

\begin{enumerate}
\item Send me an email with the URL of your GitHub repository\\
        (e.g., \code{https://github.com/username/aqmss2})
\item Requirements:

\end{enumerate}

\begin{itemize}
  \item Your repository is \textbf{public}
  \item It contains a README with your name
  \item It has a \code{data/} folder with the Gapminder CSV file
  \item It has a \code{assignment1/} folder with your R files and a saved plot
\end{itemize}

\end{document}
