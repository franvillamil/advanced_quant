% Preamble for problem sets, exams, and project descriptions
% Applied Quantitative Methods for the Social Sciences II

\documentclass[12pt, a4paper]{article}
\usepackage[margin = 2cm]{geometry}
\usepackage{graphicx}
\usepackage[english]{babel}
\usepackage[utf8]{inputenc}
\usepackage[colorlinks = TRUE, linkcolor = blue, urlcolor = blue, citecolor = blue]{hyperref}
\usepackage{setspace}
\setstretch{1.25}
\renewcommand*\rmdefault{ppl}

\usepackage[]{titlesec}
    \titleformat*{\section}{\large\bf}
    \titleformat*{\subsection}{\normalsize\bf}
    \titleformat*{\subsubsection}{\normalsize\it}

% For code blocks
\usepackage{listings}
\usepackage{xcolor}
\lstset{
    basicstyle=\ttfamily\small,
    backgroundcolor=\color{gray!10},
    frame=single,
    framerule=0pt,
    breaklines=true,
    columns=fullflexible,
    keepspaces=true
}

% For inline code
\newcommand{\code}[1]{\texttt{#1}}

% Enumerate with letters
\usepackage{enumitem}

\textbf{\large Assignment 3: Binary Outcomes}\\\vspace{10pt}
\end{center}

\vspace{10pt}
\noindent
\textbf{\large Instructions:}

\vspace{10pt}
\begin{itemize}
\setlength\itemsep{0pt}
  \item {\color{red}{\textbf{Deadline}}}: \textbf{February 26, before class}
  \item Submit your work as a \code{.R} file called \code{ps3.R} in your GitHub repository
  \item Use comments in your R code to answer conceptual questions and explain your analysis
  \item For model interpretation, use the \code{marginaleffects} package
  \item You are encouraged to work together, but each person must submit their own code
  \item Part 1 is started in class; Part 2 is completed at home
\end{itemize}

\vspace{20pt}
\tableofcontents
\newpage

% ==========================================================================
\section{Part 1: In-Class Lab (ANES Voter Turnout)}
% ==========================================================================

In this lab, we analyze voter turnout using data from the American National Election Study (ANES). You will practice estimating and comparing the linear probability model and logistic regression, computing marginal effects, and plotting predicted probabilities. Follow along in class.

\subsection{Setup and data preparation}

Download \code{anes.csv} from the course page.

\begin{enumerate}[label=\alph*)]
  \item Load the dataset. Key variables:
  \begin{itemize}
    \item \code{voted} --- whether the respondent voted (0/1, our outcome)
    \item \code{age} --- age in years
    \item \code{female} --- gender indicator (1 = female, 0 = male)
    \item \code{education} --- years of education
    \item \code{income} --- household income (standardized scale)
    \item \code{party\_id} --- strength of party identification (1--7 scale)
  \end{itemize}
  \item Drop observations with missing values on any of these variables. How many observations remain?
  \item Compute the overall turnout rate (proportion of \code{voted == 1}) and print summary statistics for all variables.
\end{enumerate}

\subsection{Exploratory visualization}

\begin{enumerate}[label=\alph*)]
  \item Create a bar chart showing the turnout rate by education level (hint: compute the mean of \code{voted} for each value of \code{education}, then use \code{geom\_col()}).
  \item In a comment, describe the pattern. Does turnout increase with education?
\end{enumerate}

\subsection{Linear probability model}

\begin{enumerate}[label=\alph*)]
  \item Estimate an LPM with \code{voted} as the outcome and \code{age}, \code{education}, \code{income}, and \code{female} as predictors: \\\code{lpm <- lm(voted \textasciitilde{} age + education + income + female, data = df)}.
  \item Print the results using \code{broom::tidy()}.
  \item Interpret the coefficient on \code{education} in a comment. What does it mean in terms of probability?
  \item Check the predicted probabilities: how many are below 0 or above 1? Report the minimum and maximum predicted values.
\end{enumerate}

\subsection{Logistic regression}

\begin{enumerate}[label=\alph*)]
  \item Estimate a logit model with the same predictors: \\\code{logit <- glm(voted \textasciitilde{} age + education + income + female,} \\\code{\hspace{60pt}family = binomial, data = df)}.
  \item Print the results using \code{broom::tidy()}.
  \item Compute the odds ratios using \code{exp(coef(logit))}. Interpret the odds ratio for \code{education} in a comment.
  \item Verify that all predicted probabilities are between 0 and 1.
\end{enumerate}

\subsection{Comparing LPM and logit}

\begin{enumerate}[label=\alph*)]
  \item Compute average marginal effects for the logit model using \code{marginaleffects::avg\_slopes(logit)}.
  \item Compare the AMEs to the LPM coefficients. How similar are they? Discuss in a comment.
  \item Create a table with \code{modelsummary()} showing the LPM and logit side by side. Use robust standard errors for the LPM: \\\code{modelsummary(list("LPM" = lpm, "Logit" = logit), vcov = list("robust", NULL))}.
\end{enumerate}

\subsection{Predicted probabilities}

\begin{enumerate}[label=\alph*)]
  \item Use \code{plot\_predictions(logit, condition = "education")} to plot the predicted probability of voting across education levels. Save the plot.
  \item Create a second plot showing predicted probabilities across \code{age} for men and women separately: \\\code{plot\_predictions(logit, condition = c("age", "female"))}.
  \item In a comment, describe the patterns. How does the effect of age differ from the effect of education?
\end{enumerate}

\subsection{Presenting results}

\begin{enumerate}[label=\alph*)]
  \item Create a coefficient plot comparing the LPM and logit models using \code{modelplot()}.
  \item Save the plot.
  \item In a comment: for this dataset, do the LPM and logit lead to different substantive conclusions? When might the differences matter?
\end{enumerate}

\newpage

% ==========================================================================
\section{Part 2: Take-Home Exercises (STAR --- High School Graduation)}
% ==========================================================================

We return to the STAR experiment data from Assignment 2, but now focus on a \textbf{binary outcome}: whether students graduated from high school. This lets you practice binary outcome methods on a dataset you already know.

Use the same \code{star.csv} from the course page. Key variables for this assignment:

\begin{itemize}
  \item \code{classtype}: class type (1 = small, 2 = regular, 3 = regular + aide)
  \item \code{race}: student race (1 = White, 2 = Black, 3 = Asian, 4 = Hispanic, 5 = Native American, 6 = Other)
  \item \code{yearssmall}: years spent in a small class (0--4)
  \item \code{hsgrad}: graduated from high school (0/1, our outcome)
\end{itemize}

\subsection{Data preparation}

\begin{enumerate}[label=\alph*)]
  \item Load \code{star.csv} and create the same factor variables as in Assignment 2: \code{classtype} with labels \code{"Small"}, \code{"Regular"}, \code{"Regular+Aide"}, and \code{race} with labels \code{"White"}, \code{"Black"}, etc.
  \item Create a binary variable \code{small} equal to 1 if the student was in a small class and 0 otherwise.
  \item Drop observations with missing values on \code{hsgrad}. How many observations remain?
  \item Compute the high school graduation rate overall and by class type. In a comment, describe the differences.
\end{enumerate}

\subsection{LPM and logit}

\begin{enumerate}[label=\alph*)]
  \item Estimate an LPM predicting \code{hsgrad} from \code{small}: \\\code{lpm1 <- lm(hsgrad \textasciitilde{} small, data = df)}.
  \item Estimate a logit model with the same predictor: \\\code{logit1 <- glm(hsgrad \textasciitilde{} small, family = binomial, data = df)}.
  \item Interpret the LPM coefficient on \code{small}: what is the estimated difference in graduation probability between small and non-small classes?
  \item Compute the AME from the logit using \code{avg\_slopes(logit1)}. How does it compare to the LPM coefficient?
\end{enumerate}

\subsection{Adding controls}

\begin{enumerate}[label=\alph*)]
  \item Estimate both LPM and logit with controls: \\\code{lpm2 <- lm(hsgrad \textasciitilde{} small + race + yearssmall, data = df)} \\\code{logit2 <- glm(hsgrad \textasciitilde{} small + race + yearssmall,} \\\code{\hspace{60pt}family = binomial, data = df)}.
  \item Compare the coefficient on \code{small} between the bivariate and controlled models. Does it change much? What does this tell you about the randomization?
  \item Interpret the coefficient on \code{yearssmall} from the logit model. Use \code{avg\_slopes()} to convert to a marginal effect.
\end{enumerate}

\subsection{Predicted probabilities}

\begin{enumerate}[label=\alph*)]
  \item Using the controlled logit model, compute predicted graduation probabilities for:
  \begin{itemize}
    \item A White student in a small class with 3 years in small classes
    \item A Black student in a regular class with 0 years in small classes
  \end{itemize}
  Use \code{predictions(logit2, newdata = datagrid(...))}. Report the estimates and 95\% CIs.
  \item Plot predicted graduation probabilities across \code{yearssmall} for small vs.\ non-small classes: \\\code{plot\_predictions(logit2, condition = c("yearssmall", "small"))}. Save the plot.
\end{enumerate}

\subsection{Interactions}

\begin{enumerate}[label=\alph*)]
  \item Does the small class effect on graduation differ by race? Estimate: \\\code{logit3 <- glm(hsgrad \textasciitilde{} small * race + yearssmall,} \\\code{\hspace{60pt}family = binomial, data = df)}.
  \item Use \code{avg\_slopes(logit3, variables = "small", by = "race")} to compute the marginal effect of \code{small} separately for each racial group.
  \item In a comment, discuss: is the small class effect larger for some groups than others?
\end{enumerate}

\subsection{Presenting results and discussion}

\begin{enumerate}[label=\alph*)]
  \item Create a table with \code{modelsummary()} comparing all four models (LPM bivariate, LPM controlled, logit bivariate, logit controlled). Use robust SEs for the LPM models.
  \item Create a coefficient plot with \code{modelplot()}.
  \item In a comment (5--10 sentences), discuss:
  \begin{itemize}
    \item What does the STAR data suggest about the effect of small class sizes on high school graduation?
    \item How do the LPM and logit results compare? Do they tell a similar or different story?
    \item Why is this experimental evidence more credible than an observational study?
  \end{itemize}
\end{enumerate}

\vspace{15pt}

\section{Data Sources}

\begin{itemize}
  \item \textbf{ANES dataset}: \code{anes.csv}, provided on the course page. A subset of the American National Election Study with demographic and political participation variables.
  \item \textbf{STAR dataset}: \code{star.csv}, provided on the course page. Same data as Assignment 2, from the Tennessee STAR experiment (Krueger, 1999).
\end{itemize}

\vspace{15pt}

\section{Submission}

Commit your \code{ps3.R} file to your GitHub repository before the deadline. Make sure your repository is public so I can access it.

Your R script should:
\begin{itemize}
  \item Be well-organized with clear section headers (using comments)
  \item Include all code needed to reproduce your analysis
  \item Include your answers and interpretations as comments
  \item Save any plots to files (e.g., using \code{ggsave()})
  \item Run without errors from top to bottom
\end{itemize}

\end{document}
