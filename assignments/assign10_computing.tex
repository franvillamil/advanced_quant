% Preamble for problem sets, exams, and project descriptions
% Applied Quantitative Methods for the Social Sciences II

\documentclass[12pt, a4paper]{article}
\usepackage[margin = 2cm]{geometry}
\usepackage{graphicx}
\usepackage[english]{babel}
\usepackage[utf8]{inputenc}
\usepackage[colorlinks = TRUE, linkcolor = blue, urlcolor = blue, citecolor = blue]{hyperref}
\usepackage{setspace}
\setstretch{1.25}
\renewcommand*\rmdefault{ppl}

\usepackage[]{titlesec}
    \titleformat*{\section}{\large\bf}
    \titleformat*{\subsection}{\normalsize\bf}
    \titleformat*{\subsubsection}{\normalsize\it}

% For code blocks
\usepackage{listings}
\usepackage{xcolor}
\lstset{
    basicstyle=\ttfamily\small,
    backgroundcolor=\color{gray!10},
    frame=single,
    framerule=0pt,
    breaklines=true,
    columns=fullflexible,
    keepspaces=true
}

% For inline code
\newcommand{\code}[1]{\texttt{#1}}

% Enumerate with letters
\usepackage{enumitem}

\textbf{\large Assignment 10: Best Practices in Computing}\\\vspace{10pt}
\end{center}

\vspace{10pt}
\noindent
\textbf{\large Instructions:}

\vspace{10pt}
\begin{itemize}
\setlength\itemsep{0pt}
  \item {\color{red}{\textbf{Deadline}}}: \textbf{April 23, before class}
  \item Submit via your GitHub repository in a separate folder (\texttt{assignment10})
  \item \textbf{Always use comments} in your R code -- and use them to answer questions
  \item You are encouraged to work together, but each person must submit their own code
  \item Plan is to start Part 1 in class and complete Part 2 at home
  \item I'll upload a solution file to the website after next class
\end{itemize}

\vspace{20pt}
\tableofcontents
\newpage

% ==========================================================================
\section{Part 1: In-Class (Project Organization)}
% ==========================================================================

{\sloppy In this lab, you will create a properly organized mini-project that follows the \texttt{workflow\_example} pattern from the lecture. Recall the key principle: separate tasks into folders, each with its own script and output subfolder. You will use the \code{corruption.dta} dataset from Assignment~4 to build a complete, reproducible pipeline.\par}

\subsection{Folder structure}

Create a new folder \texttt{assignment10} in your course repository with the following structure:

\begin{lstlisting}
assignment10/
  Makefile
  README.md
  data/
    corruption.dta
  analysis/
    models.R
    output/
  plots/
    figures.R
    output/
\end{lstlisting}

\begin{enumerate}[label=\alph*)]
  \item Create all the folders and empty files listed above. You can do this from the terminal using \code{mkdir} and \code{touch}, or manually.
  \item Download the \code{corruption.dta} dataset and place it in the \code{data/} subfolder. The dataset is available at:
  \begin{itemize}
    \item \href{https://github.com/franvillamil/AQM2/tree/master/datasets/other}{github.com/franvillamil/AQM2/tree/master/datasets/other}
  \end{itemize}
  \item In \code{README.md}, write a brief description of the project and what each folder contains (3--5 sentences). For example: what question does the analysis address? What does the \code{data/} folder contain? What do the \code{analysis/} and \code{plots/} folders produce?
\end{enumerate}

\subsection{Analysis script}

Create the file \code{analysis/models.R} with the following structure. The script should be self-contained: anyone should be able to run it from the \texttt{assignment10} folder and reproduce the results.

\begin{enumerate}[label=\alph*)]
  \item Start the script by loading necessary packages and reading the data:
  \begin{lstlisting}
# setwd("~/path/to/assignment10")
options(stringsAsFactors = FALSE)

library(readstata13)
library(modelsummary)

df = read.dta13("data/corruption.dta")
  \end{lstlisting}
  \item Define constants at the top of the script for the key variables:
  \begin{lstlisting}
dep_var = "ti_cpi"
indep_var = "undp_gdp"
  \end{lstlisting}
  \item Drop observations with missing values on these two variables. Then print the number of observations after cleaning:
  \begin{lstlisting}
df = df[!is.na(df[[dep_var]]) & !is.na(df[[indep_var]]), ]
cat("Observations:", nrow(df), "\n")
  \end{lstlisting}
  \item Add an assertion after cleaning to check that the dataset is not unexpectedly small:
  \begin{lstlisting}
if(nrow(df) < 10) stop("Too few observations")
  \end{lstlisting}
  \item Estimate two models --- one in levels and one using the log of GDP:
  \begin{lstlisting}
m1 = lm(ti_cpi ~ undp_gdp, data = df)
m2 = lm(ti_cpi ~ log(undp_gdp), data = df)
  \end{lstlisting}
  \item Use \code{modelsummary()} to save a \LaTeX{} table to \code{analysis/output/table\_models.tex}:
  \begin{lstlisting}
modelsummary(
  list("Level" = m1, "Log" = m2),
  output = "analysis/output/table_models.tex",
  stars = TRUE,
  gof_map = c("r.squared", "nobs"))
  \end{lstlisting}
  \item Print a message when the script completes: \\\code{cat("Analysis complete. Table saved.\textbackslash n")}
\end{enumerate}

\subsection{Plots script}

Create the file \code{plots/figures.R}. This script reads the same data and produces a scatter plot.

\begin{enumerate}[label=\alph*)]
  \item Load packages (\code{readstata13}, \code{ggplot2}), read the data, and drop missing values.
  \item Create a scatter plot of corruption (y-axis) vs.\ log GDP (x-axis) using \code{geom\_point()}. Add a linear fit with \code{geom\_smooth()}.
  \item Use \code{theme\_minimal()}, add informative axis labels and a title.
  \item Save to \code{plots/output/scatter\_corruption.pdf} using \code{ggsave()}, specifying width and height explicitly.
  \item Print a message when done: \code{cat("Scatter plot saved.\textbackslash n")}
\end{enumerate}

\subsection{Makefile}

Create a \code{Makefile} in the \texttt{assignment10} folder that automates the pipeline. Recall from the lecture that a Makefile specifies targets, dependencies, and recipes.

\begin{enumerate}[label=\alph*)]
  \item Write an \code{all:} rule that lists both outputs as targets:
  \begin{lstlisting}
all: analysis/output/table_models.tex \
     plots/output/scatter_corruption.pdf
  \end{lstlisting}
  \item Write a rule for the analysis table that depends on the R script and the dataset:
  \begin{lstlisting}
analysis/output/table_models.tex: analysis/models.R \
                                  data/corruption.dta
	Rscript --no-save analysis/models.R
  \end{lstlisting}
  \textbf{Important}: the recipe line (the \code{Rscript} command) must be indented with a \textbf{tab character}, not spaces. This is a Makefile requirement.
  \item Write a similar rule for the scatter plot output, which depends on \code{plots/figures.R} and the dataset.
  \item Test it by running \code{make} in the terminal from the \texttt{assignment10} folder. In a comment in your \code{README.md}, note whether it worked and what output was produced.
\end{enumerate}

\newpage

% ==========================================================================
\section{Part 2: Take-Home (Code Quality and Git)}
% ==========================================================================

\subsection{Code improvement}

The following R script works but has multiple style and organization problems. Your task is to rewrite it following the best practices from the lecture (DRY, constants, meaningful names, comments, assertions).

\begin{lstlisting}
x=read.csv("mydata.csv")
x=x[x$year>=2000,]
x=x[x$year<=2020,]
m=lm(x$outcome~x$gdp+x$pop)
summary(m)
m2=lm(x$outcome~x$gdp+x$pop+x$education)
summary(m2)
m3=lm(x$outcome~x$gdp+x$pop+x$education+x$health)
summary(m3)
pdf("fig.pdf")
plot(x$gdp,x$outcome)
dev.off()
pdf("fig2.pdf")
plot(x$education,x$outcome)
dev.off()
\end{lstlisting}

Rewrite this script and save it as \code{assignment10/improved\_script.R}. Your improved version should include:

\begin{enumerate}[label=\alph*)]
  \item Meaningful variable names instead of \code{x}, \code{m}, \code{m2}, \code{m3}.
  \item Constants defined at the top of the script for the year range (e.g., \code{start\_year = 2000}, \code{end\_year = 2020}).
  \item Comments explaining what each section of the code does, with section dividers (\code{\# ============}).
  \item A function to avoid repeating the \code{pdf()}/\code{plot()}/\code{dev.off()} pattern. For example:
  \begin{lstlisting}
save_scatter = function(data, xvar, yvar, filename) {
  pdf(filename, width = 7, height = 5)
  plot(data[[xvar]], data[[yvar]],
       xlab = xvar, ylab = yvar)
  dev.off()
}
  \end{lstlisting}
  \item An assertion after loading the data (e.g., check that the number of rows is greater than zero, or check that the expected columns exist).
  \item Proper formatting: spaces around operators, consistent indentation, use of the \code{data =} argument in \code{lm()} instead of the \code{\$} notation.
  \item A diagnostic print statement after filtering, e.g.:
  \begin{lstlisting}
cat("Rows after filtering:", nrow(df), "\n")
  \end{lstlisting}
\end{enumerate}

\subsection{Git practices}

This exercise asks you to practice the Git habits discussed in the lecture: frequent commits with meaningful messages, and a proper \code{.gitignore}.

\begin{enumerate}[label=\alph*)]
  \item Make sure your \texttt{assignment10} folder is tracked by Git (it should be part of your course repository).
  \item Create at least \textbf{3 separate commits} for this assignment. Each commit should correspond to a logical unit of work (e.g., one commit for the folder structure, one for the analysis script, one for the Makefile). Do \textbf{not} make one big commit with everything at the end.
  \item Each commit must have a \textbf{meaningful message} --- not ``update'' or ``changes''. Recall from the lecture: the message should complete the sentence ``This commit will...''. Examples:
  \begin{itemize}
    \item \code{Add folder structure and README for assignment 10}
    \item \code{Create analysis script with regression models}
    \item \code{Add Makefile to automate pipeline}
  \end{itemize}
  \item Create a \code{.gitignore} file inside \texttt{assignment10/} that ignores generated and system files:
  \begin{lstlisting}
*.pdf
*.aux
*.log
.DS_Store
analysis/output/
plots/output/
  \end{lstlisting}
  \item Push everything to GitHub.
  \item In a comment at the end of your \code{README.md}, list the commit messages you used for this assignment.
\end{enumerate}

\subsection{Reflection}

At the end of your \code{README.md}, answer the following questions in a few sentences each:

\begin{enumerate}[label=\alph*)]
  \item What was the most useful practice you learned from the computing lecture?
  \item How would you apply these practices to your final project for this course?
  \item What is one thing from the ``bad script'' example (in the code improvement exercise) that you recognized from your own coding habits?
\end{enumerate}

\vspace{15pt}

% ==========================================================================
\section{Data Sources}
% ==========================================================================

The corruption dataset is available at the course GitHub repository:
\begin{itemize}
  \item \href{https://github.com/franvillamil/AQM2/tree/master/datasets/other}{github.com/franvillamil/AQM2/tree/master/datasets/other} (\code{corruption.dta})
\end{itemize}

\vspace{15pt}

% ==========================================================================
\section{Submission}
% ==========================================================================

Commit your work to your GitHub repository before the deadline. Everything should be inside the \texttt{assignment10} folder. Make sure your repository is public so I can access it.

Your \texttt{assignment10} folder should contain:
\begin{itemize}
  \item \code{README.md} --- project description, Makefile test results, commit messages, and reflection
  \item \code{Makefile} --- automation rules for the analysis and plots
  \item \code{.gitignore} --- ignoring generated output and system files
  \item \code{data/corruption.dta} --- the dataset
  \item \code{analysis/models.R} --- regression analysis script
  \item \code{analysis/output/} --- (generated table, may be git-ignored)
  \item \code{plots/figures.R} --- scatter plot script
  \item \code{plots/output/} --- (generated figures, may be git-ignored)
  \item \code{improved\_script.R} --- rewritten version of the bad script
\end{itemize}

All R scripts should run without errors from the \texttt{assignment10} directory.

\end{document}
