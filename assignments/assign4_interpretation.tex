% Preamble for problem sets, exams, and project descriptions
% Applied Quantitative Methods for the Social Sciences II

\documentclass[12pt, a4paper]{article}
\usepackage[margin = 2cm]{geometry}
\usepackage{graphicx}
\usepackage[english]{babel}
\usepackage[utf8]{inputenc}
\usepackage[colorlinks = TRUE, linkcolor = blue, urlcolor = blue, citecolor = blue]{hyperref}
\usepackage{setspace}
\setstretch{1.25}
\renewcommand*\rmdefault{ppl}

\usepackage[]{titlesec}
    \titleformat*{\section}{\large\bf}
    \titleformat*{\subsection}{\normalsize\bf}
    \titleformat*{\subsubsection}{\normalsize\it}

% For code blocks
\usepackage{listings}
\usepackage{xcolor}
\lstset{
    basicstyle=\ttfamily\small,
    backgroundcolor=\color{gray!10},
    frame=single,
    framerule=0pt,
    breaklines=true,
    columns=fullflexible,
    keepspaces=true
}

% For inline code
\newcommand{\code}[1]{\texttt{#1}}

% Enumerate with letters
\usepackage{enumitem}

\textbf{\large Assignment 4: Model Interpretation and Diagnostics}\\\vspace{10pt}
\end{center}

\vspace{10pt}
\noindent
\textbf{\large Instructions:}

\vspace{10pt}
\begin{itemize}
\setlength\itemsep{0pt}
  \item {\color{red}{\textbf{Deadline}}}: \textbf{March 5, before class}
  \item Submit your work in a separate folder in your GitHub repository
  \begin{itemize}
    \item You can include only the R file or additional ones (e.g. pdf with results)
  \end{itemize}
  \item \textbf{Always use comments} in your R code -- and use them to answer questions
  \item You are encouraged to work together, but each person must submit their own code
  \item Plan is to start Part 1 in class and complete Part 2 at home
  \item I'll upload a solution file to the website after next class
\end{itemize}

\vspace{20pt}
\tableofcontents
\newpage

% ==========================================================================
\section{Part 1: In-Class (Corruption and Wealth)}
% ==========================================================================

In this lab, we analyze the relationship between corruption and economic development using cross-country data. You will practice computing predicted values, marginal effects, and creating publication-quality tables and plots. Recall from the lecture that raw coefficients are rarely enough --- we need to compute \textbf{quantities of interest} to communicate results effectively.

\subsection{Setup and data exploration}

Download the data here:
\begin{itemize}
  \item \href{https://github.com/franvillamil/AQM2/tree/master/datasets/other}{github.com/franvillamil/AQM2/tree/master/datasets/other}
\end{itemize}

\begin{enumerate}[label=\alph*)]
  \item Load the \code{corruption.dta} dataset using \code{readstata13::read.dta13()}. Key variables:
  \begin{itemize}
    \item \code{cname} --- country name
    \item \code{ti\_cpi} --- Corruption Perceptions Index (0--10 scale, higher = less corrupt)
    \item \code{undp\_gdp} --- GDP per capita (PPP, in dollars)
  \end{itemize}
  \item Drop observations with missing values on \code{ti\_cpi} or \code{undp\_gdp}. How many countries remain?
  \item Compute summary statistics for \code{ti\_cpi} and \code{undp\_gdp}. In a comment, note the range and standard deviation of each variable. Is GDP per capita right-skewed?
\end{enumerate}

\subsection{Exploratory visualization}

\begin{enumerate}[label=\alph*)]
  \item Create a scatter plot of \code{ti\_cpi} (y-axis) against \code{undp\_gdp} (x-axis) using \code{geom\_point()}. Add a smooth line with \code{geom\_smooth(method = "lm")}.
  \item In a comment, describe the pattern. Does the relationship look linear?
  \item Now create a second scatter plot with \code{log(undp\_gdp)} on the x-axis. Does the log transformation improve the linearity of the relationship?
\end{enumerate}

\subsection{Bivariate regression}

\begin{enumerate}[label=\alph*)]
  \item Estimate a bivariate regression of corruption on GDP per capita: \\\code{m1 = lm(ti\_cpi \textasciitilde{} undp\_gdp, data = df)}.
  \item Print the results using \code{summary()} or \code{broom::tidy()}. In a comment, interpret the coefficient on \code{undp\_gdp}. What is the predicted change in the corruption index for a \$10,000 increase in GDP per capita?
  \item Compute the 25th and 75th percentiles of GDP per capita using \code{quantile()}.
  Then use \code{predictions()} to get predicted corruption scores at these values:
  \begin{lstlisting}
library(marginaleffects)
predictions(m1, newdata = datagrid(undp_gdp = c(q25, q75)))
  \end{lstlisting}
  In a comment, report the predicted values and their 95\% confidence intervals. What is the difference in predicted corruption between a country at the 25th percentile vs.\ the 75th percentile of GDP?
\end{enumerate}

\subsection{Non-linear specifications}

\begin{enumerate}[label=\alph*)]
  \item Estimate a model using the log of GDP per capita: \\\code{m2 = lm(ti\_cpi \textasciitilde{} log(undp\_gdp), data = df)}.
  \item Interpret the coefficient on \code{log(undp\_gdp)}. You can infer this from the mathematical properties of level-log models, but the most efficient way is to create a prediction plot.
  \item Estimate a model with a quadratic GDP term: \\\code{m3 = lm(ti\_cpi \textasciitilde{} undp\_gdp + I(undp\_gdp\^{}2), data = df)}.
  \item Compare the $R^2$ of all three models. Which specification fits the data best? In a comment, explain why a non-linear specification might be appropriate for this relationship.
\end{enumerate}

\subsection{Marginal effects}

\begin{enumerate}[label=\alph*)]
  \item For the log model (\code{m2}), compute the average marginal effect of GDP using: \\\code{avg\_slopes(m2, variables = "undp\_gdp")}.
  \item In a comment, explain why the AME differs from the raw coefficient on \code{log(undp\_gdp)}. What does the AME tell you in substantive terms?
  \item For the quadratic model (\code{m3}), compute marginal effects at specific GDP values using:
  \begin{lstlisting}
slopes(m3, variables = "undp_gdp",
       newdata = datagrid(undp_gdp = c(2000, 10000, 30000)))
  \end{lstlisting}
  In a comment, describe how the marginal effect of GDP changes as countries become richer. Does the effect diminish?
\end{enumerate}

\subsection{Prediction plots}

\begin{enumerate}[label=\alph*)]
  \item Create a prediction plot for the log model: \\\code{plot\_predictions(m2, condition = "undp\_gdp")}. Save the plot.
  \item Create a prediction plot for the quadratic model (\code{m3}) on the same variable. Save this plot too.
  \item In a comment, compare the two plots. Do the models tell a similar story about the corruption--wealth relationship? Where do they diverge?
\end{enumerate}

\subsection{Residual diagnostics}

\begin{enumerate}[label=\alph*)]
  \item Use \code{broom::augment(m1)} to get residuals and fitted values from the level-level model. Create a scatter plot of residuals (\code{.resid}) vs.\ fitted values (\code{.fitted}). Does the plot suggest non-linearity or heteroskedasticity?
  \item Now do the same for the log model (\code{m2}). Does the log transformation improve the residual pattern?
  \item Identify influential observations using Cook's distance. Use \code{plot(m2, which = 4)} or compute Cook's distance manually with \code{cooks.distance(m2)}. Which countries (if any) have Cook's distance above $4/n$? Look up their names.
  \item In a comment, discuss: should these influential observations be removed? What would you recommend as a robustness check?
\end{enumerate}

\subsection{Publication-quality table}

\begin{enumerate}[label=\alph*)]
  \item Create a regression table comparing all three models using:
  \begin{lstlisting}
library(modelsummary)
modelsummary(
  list("Level-Level" = m1, "Level-Log" = m2, "Quadratic" = m3),
  vcov = "robust",
  stars = TRUE,
  gof_map = c("r.squared", "nobs"))
  \end{lstlisting}
  \item In a comment, summarize: which model would you choose for a final presentation, and why?
\end{enumerate}

\newpage

% ==========================================================================
\section{Part 2: Take-Home (Wealth and Infant Mortality)}
% ==========================================================================

We now turn to another cross-country question: the relationship between national income and infant mortality. This exercise asks you to build and compare multiple specifications, compute predicted values for specific scenarios, and create a publication-quality visualization.

Download \code{infantmortality.dta} from:
\begin{itemize}
  \item \href{https://github.com/franvillamil/AQM2/tree/master/datasets/other}{github.com/franvillamil/AQM2/tree/master/datasets/other}
\end{itemize}

Key variables:
\begin{itemize}
  \item \code{country} --- country name
  \item \code{region} --- world region (Africa, Americas, Asia, Europe)
  \item \code{income} --- per-capita income (dollars)
  \item \code{infant} --- infant mortality rate (per 1,000 live births)
  \item \code{oil} --- oil-exporting country (yes/no)
\end{itemize}

\subsection{Data exploration}

\begin{enumerate}[label=\alph*)]
  \item Load the dataset and print summary statistics for all variables. How many countries are in the data?
  \item Create a histogram of \code{infant} and a histogram of \code{income}. Are either of them right-skewed?
  \item Create a scatter plot of \code{infant} (y-axis) against \code{income} (x-axis), coloring points by \code{region}. Describe the relationship in a comment.
  \item Create the same scatter plot but using \code{log(income)} on the x-axis and \code{log(infant)} on the y-axis. Does the log-log relationship look more linear?
\end{enumerate}

\subsection{Comparing specifications}

\begin{enumerate}[label=\alph*)]
  \item Estimate a level-level model: \\\code{m1 = lm(infant \textasciitilde{} income, data = df)}.
  \item Estimate a log-log model: \\\code{m2 = lm(log(infant) \textasciitilde{} log(income), data = df)}.
  \item Interpret the coefficient on income in each model:
  \begin{itemize}
    \item In \code{m1}: what is the predicted change in infant mortality for a \$1,000 increase in income?
    \item In \code{m2}: recall that the log-log coefficient is an \textbf{elasticity}. What does it mean here? (e.g., ``A 10\% increase in income is associated with a \_\% change in infant mortality.'')
  \end{itemize}
  \item Create a residuals vs.\ fitted values plot for both models. Which specification has a better residual pattern? Discuss in a comment.
\end{enumerate}

\subsection{Multiple regression with controls}

\begin{enumerate}[label=\alph*)]
  \item Estimate a log-log model with controls for region and oil-exporting status: \\\code{m3 = lm(log(infant) \textasciitilde{} log(income) + region + oil, data = df)}.
  \item Print the results. In a comment, interpret the coefficient on \code{log(income)}: does controlling for region and oil status change the income effect?
  \item Interpret the coefficient on the Africa region indicator (relative to the reference category). What does it tell you about infant mortality in Africa, controlling for income?
  \item Compute average marginal effects using \code{avg\_slopes(m3)}. Focus on the AME of \code{income} and report it in a comment.
\end{enumerate}

\subsection{Interaction: oil status and income}

\begin{enumerate}[label=\alph*)]
  \item Estimate a model with an interaction between oil status and log income: \\\code{m4 = lm(log(infant) \textasciitilde{} log(income) * oil + region, data = df)}.
  \item Use \code{avg\_slopes(m4, variables = "income", by = "oil")} to compute the marginal effect of income separately for oil-exporting and non-oil countries.
  \item In a comment, discuss: does the relationship between income and infant mortality differ for oil-exporting countries? What might explain this?
  \item Plot how the marginal effect of income varies by oil status: \\\code{plot\_slopes(m4, variables = "income", condition = "oil")}. Save the plot.
\end{enumerate}

\subsection{Predicted values for specific scenarios}

\begin{enumerate}[label=\alph*)]
  \item Using model \code{m3} (without interaction), compute predicted infant mortality rates for:
  \begin{itemize}
    \item A non-oil African country with income = \$1,000
    \item A non-oil European country with income = \$20,000
    \item An oil-exporting country in the Americas with income = \$10,000
  \end{itemize}
  Use:
  \begin{lstlisting}
predictions(m3,
  newdata = datagrid(
    income = c(1000, 20000, 10000),
    region = c("Africa", "Europe", "Americas"),
    oil = c("no", "no", "yes")))
  \end{lstlisting}
  Note: since the outcome is \code{log(infant)}, you need to exponentiate the predictions to get infant mortality in the original scale. Use \code{exp()} on the \code{estimate} column.
  \item In a comment, discuss the predicted values. Are they plausible? How large is the gap between the African and European scenarios?
\end{enumerate}

\subsection{Publication-quality visualization}

\begin{enumerate}[label=\alph*)]
  \item Create a prediction plot showing predicted infant mortality across income levels, separately by region:
  \begin{lstlisting}
plot_predictions(m3, condition = c("income", "region"))
  \end{lstlisting}
  Customize the plot to make it suitable for a general audience: add informative axis labels, a title, and use \code{theme\_minimal()} or similar. Save the plot.
  \item In a comment (5--10 sentences), discuss: what does this plot tell a general audience about the relationship between wealth and infant mortality? What role does geography play? What are the main limitations of this analysis (e.g., omitted variables, reverse causality, ecological fallacy)?
\end{enumerate}

\subsection{Diagnostics and robust inference}

\begin{enumerate}[label=\alph*)]
  \item Create a residuals vs.\ fitted values plot for \code{m3}. Does the plot suggest heteroskedasticity?
  \item Create a regression table comparing all four models with robust standard errors:
  \begin{lstlisting}
modelsummary(
  list("Level" = m1, "Log-Log" = m2,
       "Controls" = m3, "Interaction" = m4),
  vcov = "robust",
  stars = TRUE,
  gof_map = c("r.squared", "nobs"))
  \end{lstlisting}
  \item Compare the robust and default standard errors for \code{m3}. Run \code{modelsummary()} with and without \code{vcov = "robust"}. Do the conclusions change? Why use robust SEs?
\end{enumerate}

\vspace{15pt}

% ==========================================================================
\section{Data Sources}
% ==========================================================================

Both datasets are available at the course GitHub repository:
\begin{itemize}
  \item Corruption data: \href{https://github.com/franvillamil/AQM2/tree/master/datasets/other}{github.com/franvillamil/AQM2/tree/master/datasets/other}\\(\code{corruption.dta})
  \item Infant mortality data: same folder (\code{infantmortality.dta})
\end{itemize}

\vspace{15pt}

% ==========================================================================
\section{Submission}
% ==========================================================================

Commit your file to your GitHub repository before the deadline. Put it in a different folder, e.g. \texttt{assignment4}. Make sure your repository is public so I can access it.

Your R script should:
\begin{itemize}
  \item Be well-organized with clear section headers (using comments)
  \item Include all code needed to reproduce your analysis
  \item Include your answers and interpretations as comments
  \item Save any plots to files (e.g., using \code{ggsave()})
  \item Run without errors from top to bottom
\end{itemize}

\end{document}
