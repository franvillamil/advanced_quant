% ----------------------------------------------------
\begin{frame}
  \titlepage
\end{frame}
% ----------------------------------------------------
\note{Welcome everyone. Introduce yourself briefly. Mention this is the second part of the quantitative methods sequence -- they already have the foundations from AQMSS-I, now we build on that.}

% ----------------------------------------------------
\begin{frame}
\frametitle{Course overview}
\centering

\begin{itemize}[<+->]
\item This is the second part of the quantitative methods sequence
\item Focus on \textbf{applying} statistical tools in practice
\item Less theory, more hands-on work with data
\item Goal: go from research question to answer
\end{itemize}

\end{frame}
% ----------------------------------------------------
\note{Emphasize the contrast with AQMSS-I: less abstract theory, more applied work. The goal is that by the end they can take a research question, find or collect data, choose an appropriate method, and present results. Mention that we will also spend time on computing skills and workflow.}

% ----------------------------------------------------
\begin{frame}
\frametitle{What will you learn?}
\centering

\begin{itemize}[<+->]
\item How to choose the right model for your question
\item How to interpret and visualize model results
\item How to evaluate whether a model is appropriate
\item How to work with different types of data (panel, spatial, etc.)
\item Best practices in computing and reproducibility
\end{itemize}

\end{frame}
% ----------------------------------------------------
\note{Go through each point briefly. Stress that ``choosing the right model'' means matching the model to the question and the data -- not just running the fanciest thing available. Panel and spatial data are new types they probably haven't seen. Computing and reproducibility will be a recurring theme.}

% ----------------------------------------------------
\begin{frame}
\frametitle{Course structure}
% \centering

\begin{tabular}{ll}
  \textbf{Feb 5} & \asher{Introduction} \\
  \textbf{Feb 12} & \BGyellow<2>{Applied regression} \\
  \textbf{Feb 19} & \BGyellow<3>{Applied regression II (binary)} \\
  \textbf{Feb 26} & \BGyellow<4>{Interpretation and diagnostics} \\
  \textbf{Mar 5} & \BGyellow<5>{Best practices in computing} \only<5>{\textit{\footnotesize (move just before break?)}} \\
  \textbf{Mar 12} & \BGyellow<6>{Panel data I} \\
  \textbf{Mar 19} & \BGyellow<6>{Panel data II} \\
  \textbf{Mar 26} & \BGyellow<7>{Spatial data} \\
  \textit{Easter break} & \\
  \textbf{Apr 9} & \BGyellow<7>{Spatial data} \\
  \textbf{Apr 16} & \BGyellow<8>{Other outcomes} \\
  \textbf{Apr 23} & \BGyellow<9>{Project presentations} \\
  \textbf{Apr 30} & \BGyellow<10>{Exam + Review} \\
\end{tabular}

\end{frame}
% ----------------------------------------------------
\note{Walk through the schedule session by session. Mention that the order might change slightly (e.g.\ the computing session could move). Highlight that the course alternates between learning new models and applying them in problem sets. Note the Easter break and that project presentations happen right after.}

% ----------------------------------------------------
\begin{frame}
\frametitle{Evaluation}
\centering

\begin{itemize}
\item Problem sets (20\%)
  \begin{itemize}
  \item Started in class, finished at home
  \item Short deadlines
  \end{itemize}
\item Proposal presentation and peer review (10\% + 10\%)
\item Final essay (30\%)
  \begin{itemize}
  \item Small research note (max 3,000 words)
  \item Original data analysis using R
  \end{itemize}
\item Exam (30\%)
\end{itemize}

\end{frame}
% ----------------------------------------------------
\note{Problem sets are mostly graded on completion -- the point is to practice, not to get everything perfect. The final essay is the main individual assessment: they pick a question, find data, and do an original analysis. Emphasize that the proposal presentation is also a chance to get feedback from peers before writing. The exam will cover both concepts and applied interpretation.}

% ====================================================
\section{The Big Picture}
% ====================================================

% ----------------------------------------------------
\begin{frame}
\frametitle{The research process}
\centering

\vspace{10pt}

\Large
\textbf{Theory} $\longleftrightarrow$ \textbf{Data Generating Process} $\longleftrightarrow$ \textbf{Data}

\vspace{20pt}

\normalsize
\begin{itemize}[<+->]
\item Theories make claims about how the world works
\item These claims imply certain patterns in data
\item We observe data and try to learn about the underlying process
\item Our research strategy connects theory to data
\end{itemize}

\end{frame}
% ----------------------------------------------------
\note{This is the key conceptual diagram for the course. Spend some time here. Theory tells us what the world should look like; the DGP is the actual mechanism generating the data we observe; data is what we get to see. The double arrows mean we go back and forth: theory informs what patterns to expect, and data helps us refine our theories. The last point (research strategy connects theory to data) sets up the next slide.}

% ----------------------------------------------------
\begin{frame}
\frametitle{Theory first, methods second}
\centering

\begin{itemize}[<+->]
\item The research question and theory should drive everything:
  \begin{itemize}
  \item What unit of analysis to use
  \item What variation to look at
  \item What empirical strategy to follow
  \end{itemize}
\item Methods are tools to implement that strategy
\item Common mistake: picking a method and then looking for a question
\item In this course: we learn tools, but always ask \textit{why this tool for this question?}
\end{itemize}

\end{frame}
% ----------------------------------------------------
\note{This is a crucial point, especially for students who might be tempted to learn a fancy method and then look for a problem to apply it to. Give a concrete example: if your theory is about individual-level behavior, you need individual-level data and a strategy that captures individual variation -- running a country-level regression with a cool estimator doesn't help. Connect to the research design course: the sequence is always question, theory, strategy, then method. This course teaches the tools, but the question ``why this tool?'' should always come first.}

% ----------------------------------------------------
\begin{frame}
\frametitle{What is a Data Generating Process (DGP)?}
\centering

\begin{itemize}[<+->]
\item The rules that govern how data comes to exist
\item Includes:
  \begin{itemize}
  \item The social or political process we study
  \item How observations end up in our dataset
  \end{itemize}
\item We never observe the DGP directly
\item We use statistical models to make inferences about it
\end{itemize}

\end{frame}
% ----------------------------------------------------
\note{Make it concrete. Example: civil war onset. The DGP includes the actual political, economic, and social factors that lead to conflict, but also how conflicts get recorded (coding decisions, media coverage, threshold definitions). We never see the ``true'' process -- we see a dataset with rows and columns that reflect both the real process and a lot of measurement and selection choices. That's why understanding the DGP matters for choosing the right model.}

% ----------------------------------------------------
\begin{frame}
\frametitle{Why do we need statistics?}
\centering

\begin{itemize}[<+->]
\item Our theories deal with processes, not just data
\item Data is a window into the underlying process
\item Statistics helps us:
  \begin{itemize}
  \item Separate signal from noise
  \item Quantify uncertainty
  \item Make valid inferences
  \end{itemize}
\end{itemize}

\end{frame}
% ----------------------------------------------------
\note{Bridge from the DGP to methodology. We can't just look at raw data and draw conclusions -- data is noisy, incomplete, and shaped by the DGP. Statistics gives us a principled way to learn from data despite this. The three functions (signal vs noise, uncertainty, inference) will come up throughout the course: every model we fit is trying to do these things.}

% ----------------------------------------------------
\begin{frame}
\frametitle{Sources of uncertainty}
\centering

\begin{itemize}[<+->]
\item \textbf{Sampling uncertainty}: We observe a sample, not the population
\item \textbf{Theoretical uncertainty}: Our theories are simplifications
\item \textbf{Fundamental uncertainty}: Some processes are inherently random
\item[]
\item All of these create ``noise'' in our data
\item Statistical models help us deal with this noise
\end{itemize}

\end{frame}
% ----------------------------------------------------
\note{Give examples for each type. Sampling: we survey 1,000 people but want to say something about millions. Theoretical: our model of voting assumes a few variables matter but the real process is much more complex. Fundamental: even if we knew everything, some outcomes have inherent randomness (e.g.\ election margins in very close races). The key takeaway: noise is unavoidable, so we need tools that account for it rather than pretending our data is perfect.}

% ----------------------------------------------------
\begin{frame}
\frametitle{The logic of statistical inference}
\centering

\vspace{10pt}

\begin{itemize}[<+->]
\item \textbf{Probability theory}: Given a known process, what data will we see?
\item[]
\item \textbf{Statistical inference}: Given observed data, what can we learn about the process?
\item[]
\item We're doing the reverse: from data back to process
\end{itemize}

\end{frame}
% ----------------------------------------------------
\note{This is a good place to pause and check understanding. Probability theory is the ``forward'' problem: if I know the coin is fair, I can predict roughly 50\% heads. Statistical inference is the ``inverse'' problem: I see 53 heads out of 100 flips, is the coin fair? This inversion is what makes statistics hard -- there's always uncertainty about the answer. Everything we do in this course (confidence intervals, hypothesis tests, model selection) is about managing that uncertainty.}

% ====================================================
\section{Workflow Basics}
% ====================================================

% ----------------------------------------------------
\begin{frame}
\frametitle{Learning to use computers as tools}
\centering

\begin{itemize}[<+->]
\item World of quantitative methods is changing fast
\begin{itemize}
  \item e.g. Claude Code
\end{itemize}
\item I think it'll be more important to be really literate with computers
\item Part of this course will also involve learning how to properly use computers
\begin{itemize}
  \item Not using only RStudio, R Markdown, etc, but being ready to do big data-based projects
\end{itemize}
\item We'll have a session on computing, project management, etc -- but today, some notes on version control
\end{itemize}

\end{frame}
% ----------------------------------------------------
\note{Set the scene for why we're spending time on computing. AI tools like Claude Code are changing the workflow fast -- students who are comfortable with the command line, file systems, and version control will adapt much more easily. This isn't about becoming programmers, it's about being literate enough to use these tools effectively for research. Mention that we'll dedicate a full session later to project management and reproducibility, but version control is so fundamental that we start with it today.}

% ----------------------------------------------------
\begin{frame}
\frametitle{The problem: managing files over time}
\centering

\begin{itemize}[<+->]
\item Have you ever had files like this?
  \begin{itemize}
  \item \texttt{thesis\_v1.docx}
  \item \texttt{thesis\_v2\_final.docx}
  \item \texttt{thesis\_v2\_final\_REAL.docx}
  \item \texttt{thesis\_v2\_final\_REAL\_submitted.docx}
  \end{itemize}
\item[]
\item What changed between versions?
\item Which version has the correct analysis?
\item How do you collaborate without overwriting each other's work?
\end{itemize}

\end{frame}
% ----------------------------------------------------
\note{This usually gets a laugh -- everyone recognizes themselves. Let them share their own horror stories if they want. The point is that this ``system'' of naming files is fragile, unscalable, and doesn't work for collaboration at all. Ask: how would you collaborate on a data analysis project with a co-author using this approach?}

% ----------------------------------------------------
\begin{frame}
\frametitle{Version control: a better way}
\centering

\vspace{10pt}

\textbf{Version control} is a system that records changes to files over time

\vspace{15pt}

\begin{itemize}[<+->]
\item One file, complete history
\item Every change is recorded with a description
\item Can go back to any previous state
\item Multiple people can work simultaneously
\end{itemize}

\end{frame}
% ----------------------------------------------------
\note{Introduce the concept without jargon. The key idea is that instead of saving multiple copies, you save ``snapshots'' of your project over time, each with a description of what changed. You can always go back. This is like track changes in Word but much more powerful and for any kind of file.}

% ----------------------------------------------------
\begin{frame}
\frametitle{Why version control for research?}
\centering

\begin{itemize}[<+->]
\item \textbf{Reproducibility}: Track exactly what you did and when
\item \textbf{Backup}: Your work is safely stored, even if your laptop dies
\item \textbf{Collaboration}: Work with others without email chains of files
\item \textbf{Transparency}: Share your code with the research community
\item[]
\item Many journals now require or encourage sharing code via GitHub
\end{itemize}

\end{frame}
% ----------------------------------------------------
\note{Connect to their future as researchers. Reproducibility is increasingly expected -- some journals won't publish without replication materials. Mention specific examples: AJPS data policy, QJE replication packages. Collaboration is also practical for their MA thesis or co-authored papers. The transparency point matters for their credibility as researchers.}

% ----------------------------------------------------
\begin{frame}
\frametitle{Git and GitHub}
\centering

\vspace{10pt}

\textbf{Git}
\begin{itemize}
\item A version control system
\item Runs locally on your computer
\item Tracks changes to files
\end{itemize}

\vspace{15pt}

\textbf{GitHub}
\begin{itemize}
\item A web platform that hosts Git repositories
\item Stores your code online
\item Enables sharing and collaboration
\end{itemize}

\end{frame}
% ----------------------------------------------------
\note{Important distinction: Git is the engine, GitHub is the garage. Git runs on your machine and tracks changes locally. GitHub is where you store and share your work online. You can use Git without GitHub, but GitHub without Git doesn't make much sense. Analogy: Git is like saving versions on your computer; GitHub is like backing them up to the cloud and making them available to others.}

% ----------------------------------------------------
\begin{frame}
\frametitle{The basic Git workflow}
\centering

\begin{enumerate}[<+->]
\item \textbf{Make changes} to your files (write code, edit text)
\item \textbf{Stage} the changes you want to save
  \begin{itemize}
  \item ``These are the files I want to include in my next snapshot''
  \end{itemize}
\item \textbf{Commit} the staged changes with a message
  \begin{itemize}
  \item A snapshot of your project at this moment
  \end{itemize}
\item \textbf{Push} your commits to GitHub
  \begin{itemize}
  \item Upload your local changes to the cloud
  \end{itemize}
\end{enumerate}

\end{frame}
% ----------------------------------------------------
\note{Walk through each step slowly. Staging is the step that confuses people most -- explain that it's like putting items on a table before packing a box. You decide what goes in the next snapshot. A commit is the snapshot itself, with a message explaining what you did and why. Pushing uploads everything to GitHub. Emphasize that if they forget to push, their work is only on their laptop.}

% ----------------------------------------------------
\begin{frame}
\frametitle{Ways to use Git}
\centering

\begin{itemize}[<+->]
\item \textbf{GitHub web interface}: Create repos, upload files, edit directly
  \begin{itemize}
  \item Simple but limited
  \end{itemize}
\item \textbf{Command line}: Most powerful and flexible
  \begin{itemize}
  \item \texttt{git add}, \texttt{git commit}, \texttt{git push}
  \end{itemize}
\item \textbf{RStudio}: Built-in Git integration
  \begin{itemize}
  \item Point-and-click interface
  \end{itemize}
\item[]
\item All do the same thing---choose what works for you
\end{itemize}

\end{frame}
% ----------------------------------------------------
\note{Reassure them: they don't need to master the command line right now. The web interface is perfectly fine for getting started. The assignment document gives step-by-step instructions for all three approaches. Encourage them to try the command line eventually because it's the most flexible, but there's no penalty for using the web interface or RStudio.}

% ----------------------------------------------------
\begin{frame}
\frametitle{Assignment 1}
\centering

\vspace{10pt}

\begin{itemize}[<+->]
\item Create a GitHub account (if you don't have one)
\item Create a \textbf{public} repository for this course
\item Set up your README and folder structure
\item Create a simple \texttt{.R} file
\item[]
\item This repository is where you'll submit all your assignments
\item Detailed instructions in the assignment document
\end{itemize}

\end{frame}
% ----------------------------------------------------
\note{Go through the assignment briefly. The main goal is just to get everyone set up with Git and GitHub -- the actual R work is minimal. Emphasize that the repository must be public so you can check their submissions. Mention they should start early in case they run into setup issues. Point them to the detailed assignment document and the appendices on Windows setup, Positron, and Sublime Text.}

% ----------------------------------------------------
\begin{frame}
\frametitle{What makes a good analysis?}
\centering

\begin{itemize}[<+->]
\item Clear research question
\item Appropriate data for the question
\item Right statistical model for the data
\item Correct interpretation of results
\item Honest about limitations and uncertainty
\end{itemize}

\end{frame}
% ----------------------------------------------------
\note{Tie back to the Big Picture section. This is the checklist for the entire course: every analysis we do should satisfy these criteria. A clear question means knowing what you're trying to learn. Appropriate data means the data actually lets you answer the question (connects to the ``theory first'' point). Right model means matching the method to the data structure. Correct interpretation means not overclaiming. Honesty about limitations is what distinguishes good research from bad.}

% ----------------------------------------------------
\begin{frame}
\frametitle{Looking ahead}
\centering

\begin{itemize}[<+->]
\item Next session: Applied regression
\item Regression as conditional expectations
\item Multiple regression and control variables
\item Interaction effects and presenting results
\end{itemize}

\end{frame}
% ----------------------------------------------------
\note{Brief preview of what's coming. Next session will be a proper deep dive into regression -- they've seen it before in AQMSS-I but now we'll focus on applying it in R and interpreting results carefully. Mention that interactions and presenting results are things they'll use constantly in their own work.}

% ----------------------------------------------------
\begin{frame}
\frametitle{For next week}
\centering

\begin{itemize}
\item Check readings if needed
\item Review your notes on OLS from AQMSS-I
\item \textbf{Finish Assignment 1}
\end{itemize}

\end{frame}
% ----------------------------------------------------
\note{Remind them to start the assignment early. The readings are optional review if they feel rusty on OLS. The most important thing is to have their GitHub set up and working before next session.}

% ----------------------------------------------------
\begin{frame}
\frametitle{}
\centering

Questions?

\end{frame}
% ----------------------------------------------------
\note{Open the floor. If no questions, can use this time to start working on Assignment 1 together -- walk them through creating a GitHub account and repository on the projector.}
