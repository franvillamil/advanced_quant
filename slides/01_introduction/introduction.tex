\documentclass[aspectratio=43]{beamer}
% \documentclass[aspectratio=169]{beamer}

% Title --------------------------------------------
\title{\Huge Introduction}
\author{Francisco Villamil}
\date{Applied Quantitative Methods II\\IC3JM, Spring 2026}

\input{../beamer_preamble.tex}

\begin{document}

\begin{frame}
  \titlepage
\end{frame}

% ----------------------------------------------------
\begin{frame}
\frametitle{Course overview}
\centering

\begin{itemize}[<+->]
\item This is the second part of the quantitative methods sequence
\item Focus on \textbf{applying} statistical tools in practice
\item Less theory, more hands-on work with data
\item Goal: go from research question to answer
\end{itemize}

\end{frame}
% ----------------------------------------------------

% ----------------------------------------------------
\begin{frame}
\frametitle{What will you learn?}
\centering

\begin{itemize}[<+->]
\item How to choose the right model for your question
\item How to interpret and visualize model results
\item How to evaluate whether a model is appropriate
\item How to work with different types of data (panel, spatial, etc.)
\item Best practices in computing and reproducibility
\end{itemize}

\end{frame}
% ----------------------------------------------------

% ----------------------------------------------------
\begin{frame}
\frametitle{Course structure}
\centering

\begin{tabular}{ll}
  \textbf{Feb 5} & \asher{Introduction} \\
  \textbf{Feb 12-19} & \BGyellow<2>{Applied regression} \\
  \textbf{Feb 26} & \BGyellow<3>{Model interpretation and diagnostics} \\
  \textbf{Mar 5} & \BGyellow<4>{Best practices in computing} \\
  \textbf{Mar 12-19} & \BGyellow<5>{Panel data} \\
  \textbf{Mar 26 \& Apr 9} & \BGyellow<6>{Spatial data} \\
  \textbf{Apr 16} & \BGyellow<7>{Other outcomes} \\
  \textbf{Apr 23} & Project presentations \\
  \textbf{Apr 30} & Advanced topics \\
\end{tabular}

\end{frame}
% ----------------------------------------------------

% ----------------------------------------------------
\begin{frame}
\frametitle{Evaluation}
\centering

\begin{itemize}
\item Problem sets (20\%)
  \begin{itemize}
  \item Started in class, finished at home
  \item Short deadlines
  \end{itemize}
\item Proposal presentation and peer review (10\% + 10\%)
\item Final essay (30\%)
  \begin{itemize}
  \item Small research note (max 3,000 words)
  \item Original data analysis using R
  \end{itemize}
\item Exam (30\%)
\end{itemize}

\end{frame}
% ----------------------------------------------------

% ----------------------------------------------------
\begin{transitionframe}
The Big Picture
\end{transitionframe}
% ----------------------------------------------------

% ----------------------------------------------------
\begin{frame}
\frametitle{The research process}
\centering

\vspace{10pt}

\Large
\textbf{Theory} $\longleftrightarrow$ \textbf{Data Generating Process} $\longleftrightarrow$ \textbf{Data}

\vspace{20pt}

\normalsize
\begin{itemize}[<+->]
\item Theories make claims about how the world works
\item These claims imply certain patterns in data
\item We observe data and try to learn about the underlying process
\end{itemize}

\end{frame}
% ----------------------------------------------------

% ----------------------------------------------------
\begin{frame}
\frametitle{What is a Data Generating Process (DGP)?}
\centering

\begin{itemize}[<+->]
\item The rules that govern how data comes to exist
\item Includes:
  \begin{itemize}
  \item The social or political process we study
  \item How observations end up in our dataset
  \end{itemize}
\item We never observe the DGP directly
\item We use statistical models to make inferences about it
\end{itemize}

\end{frame}
% ----------------------------------------------------

% ----------------------------------------------------
\begin{frame}
\frametitle{Why do we need statistics?}
\centering

\begin{itemize}[<+->]
\item Our theories deal with processes, not just data
\item Data is a window into the underlying process
\item Statistics helps us:
  \begin{itemize}
  \item Separate signal from noise
  \item Quantify uncertainty
  \item Make valid inferences
  \end{itemize}
\end{itemize}

\end{frame}
% ----------------------------------------------------

% ----------------------------------------------------
\begin{frame}
\frametitle{Sources of uncertainty}
\centering

\begin{itemize}[<+->]
\item \textbf{Sampling uncertainty}: We observe a sample, not the population
\item \textbf{Theoretical uncertainty}: Our theories are simplifications
\item \textbf{Fundamental uncertainty}: Some processes are inherently random
\item[]
\item All of these create ``noise'' in our data
\item Statistical models help us deal with this noise
\end{itemize}

\end{frame}
% ----------------------------------------------------

% ----------------------------------------------------
\begin{frame}
\frametitle{The logic of statistical inference}
\centering

\vspace{10pt}

\begin{itemize}[<+->]
\item \textbf{Probability theory}: Given a known process, what data will we see?
\item[]
\item \textbf{Statistical inference}: Given observed data, what can we learn about the process?
\item[]
\item We're doing the reverse: from data back to process
\end{itemize}

\end{frame}
% ----------------------------------------------------

% ----------------------------------------------------
\begin{transitionframe}
Review: Key Concepts from AQMSS-I
\end{transitionframe}
% ----------------------------------------------------

% ----------------------------------------------------
\begin{frame}
\frametitle{The regression model}
\centering

\vspace{10pt}

The most common tool in social science:

\vspace{10pt}

$$Y = \beta_0 + \beta_1 X + \varepsilon$$

\vspace{10pt}

\begin{itemize}[<+->]
\item $Y$: outcome we want to explain
\item $X$: explanatory variable(s)
\item $\beta$: coefficients (what we estimate)
\item $\varepsilon$: error term (what we can't explain)
\end{itemize}

\end{frame}
% ----------------------------------------------------

% ----------------------------------------------------
\begin{frame}
\frametitle{What regression tells us}
\centering

\begin{itemize}[<+->]
\item Regression estimates \textbf{conditional expectations}
\item ``What is the average $Y$ for units with a given value of $X$?''
\item[]
\item The slope $\beta_1$ tells us:
  \begin{itemize}
  \item How much $Y$ changes, on average
  \item When comparing units that differ by 1 in $X$
  \end{itemize}
\end{itemize}

\end{frame}
% ----------------------------------------------------

% ----------------------------------------------------
\begin{frame}
\frametitle{Descriptive vs. Causal interpretation}
\centering

\begin{itemize}
\item \textbf{Descriptive}: How do units with different $X$ values compare?
  \begin{itemize}
  \item ``People with more education earn more, on average''
  \end{itemize}
\item[]
\item \textbf{Causal}: What happens if we change $X$ for a given unit?
  \begin{itemize}
  \item ``If we give someone more education, they will earn more''
  \end{itemize}
\item[]
\item Same coefficient, very different claims!
\end{itemize}

\end{frame}
% ----------------------------------------------------

% ----------------------------------------------------
\begin{frame}
\frametitle{The challenge of causal inference}
\centering

\begin{itemize}[<+->]
\item Causal effects are about \textbf{counterfactuals}
\item ``What would have happened if things were different?''
\item[]
\item The problem: we can't observe counterfactuals
\item We need strategies to infer them
\item This will be a recurring theme throughout the course
\end{itemize}

\end{frame}
% ----------------------------------------------------

% ----------------------------------------------------
\begin{frame}
\frametitle{What makes a good analysis?}
\centering

\begin{itemize}[<+->]
\item Clear research question
\item Appropriate data for the question
\item Right statistical model for the data
\item Correct interpretation of results
\item Honest about limitations and uncertainty
\end{itemize}

\end{frame}
% ----------------------------------------------------

% ----------------------------------------------------
\begin{frame}
\frametitle{Looking ahead}
\centering

\begin{itemize}[<+->]
\item Next session: Applied regression in depth
\item How to set up a regression analysis
\item How to interpret coefficients correctly
\item Common pitfalls and how to avoid them
\end{itemize}

\end{frame}
% ----------------------------------------------------

% ----------------------------------------------------
\begin{frame}
\frametitle{For next week}
\centering

\begin{itemize}
\item Read Urdinez \& Cruz (2020), chapters 1-5
\item Review your notes on OLS from AQMSS-I
\item Start Problem Set 1
\item[]
\item Check Aula Global for additional materials
\end{itemize}

\end{frame}
% ----------------------------------------------------

% ----------------------------------------------------
\begin{frame}
\frametitle{}
\centering

Questions?

\end{frame}
% ----------------------------------------------------


\end{document}
