% Preamble for problem sets, exams, and project descriptions
% Applied Quantitative Methods for the Social Sciences II

\documentclass[12pt, a4paper]{article}
\usepackage[margin = 2cm]{geometry}
\usepackage{graphicx}
\usepackage[english]{babel}
\usepackage[utf8]{inputenc}
\usepackage[colorlinks = TRUE, linkcolor = blue, urlcolor = blue, citecolor = blue]{hyperref}
\usepackage{setspace}
\setstretch{1.25}
\renewcommand*\rmdefault{ppl}

\usepackage[]{titlesec}
    \titleformat*{\section}{\large\bf}
    \titleformat*{\subsection}{\normalsize\bf}
    \titleformat*{\subsubsection}{\normalsize\it}

% For code blocks
\usepackage{listings}
\usepackage{xcolor}
\lstset{
    basicstyle=\ttfamily\small,
    backgroundcolor=\color{gray!10},
    frame=single,
    framerule=0pt,
    breaklines=true,
    columns=fullflexible,
    keepspaces=true
}

% For inline code
\newcommand{\code}[1]{\texttt{#1}}

% Enumerate with letters
\usepackage{enumitem}

\textbf{\large Appendix to Assignment 1:\\\vspace{5pt}Code editors and alternatives to RStudio}\\\vspace{10pt}
\end{center}


\noindent
In this document I list the introduction to two alternatives to RStudio, in case you want to try them out (but there is no need to):

\begin{itemize}
  \item[1.] Positron is a bit better version of RStudio, but of the same nature: an integrated software (an IDE, or `integrated development environment') where you edit R code, run it, see graphs, etc. Also valid for Python.
  \item[2.] Sublime Text is a \textit{code editor}, so a more complex tool to handle plain text files in general. You can customize it for any language you use (R, Latex, Markdown, Python, etc). It is the one I use.
\end{itemize}

\tableofcontents

\clearpage

\section{Positron}

As a potential alternative to RStudio, I recommend using \textbf{Positron} (\url{https://positron.posit.co}) as your IDE. Positron is a next-generation data science IDE built by Posit (the same company behind RStudio). It is based on VS Code and designed for R and Python.

\subsection{Installing Positron}

\begin{enumerate}
  \item Download Positron from \url{https://positron.posit.co}
  \item Install it like any other application
  \item On first launch, Positron will detect your R installation automatically
\end{enumerate}

\subsection{Differences for Assignment 1}

The following table maps the RStudio instructions in this assignment to their Positron equivalents:

\vspace{10pt}

\begin{tabular}{p{0.45\textwidth} p{0.45\textwidth}}
\textbf{RStudio} & \textbf{Positron} \\
\hline
File $\rightarrow$ New Project $\rightarrow$ Version Control $\rightarrow$ Git & File $\rightarrow$ New Folder, then open the folder. Use the built-in terminal (\code{Ctrl+`}) to run \code{git clone}. \\[8pt]
Git pane (top-right) & Source Control panel in the left sidebar (branch icon), or \code{Ctrl+Shift+G} \\[8pt]
Check box next to file to stage it & Click the \code{+} icon next to the file in the Source Control panel \\[8pt]
Click ``Commit'' button & Click the checkmark icon in the Source Control panel, type your message in the text box \\[8pt]
Click ``Push'' button & Click the \code{...} menu in Source Control $\rightarrow$ Push, or use the terminal: \code{git push} \\[8pt]
Click ``History'' (clock icon) & Use the terminal: \code{git log --oneline}, or install the ``Git Graph'' extension for a visual history \\[8pt]
Files pane (bottom-right) & Explorer panel in the left sidebar (top icon), or \code{Ctrl+Shift+E} \\[8pt]
File $\rightarrow$ New File $\rightarrow$ R Script & File $\rightarrow$ New File, then select ``R File'' \\[8pt]
Tools $\rightarrow$ Global Options $\rightarrow$ Git/SVN & Git is detected automatically. If not, open Settings (\code{Ctrl+,}) and search for ``git.path'' \\
\end{tabular}

\subsection{Key Advantages of Positron}

\begin{itemize}
  \item \textbf{Integrated terminal}: Positron has a built-in terminal (\code{Ctrl+`}) where you can run Git commands directly---no need to switch to a separate application
  \item \textbf{Better Git integration}: The Source Control panel shows diffs, staged changes, and commit history in one place
  \item \textbf{Extensions}: You can install extensions (e.g., ``Git Graph'' for visual commit history, ``R'' for enhanced R support) from the Extensions panel
  \item \textbf{Multiple languages}: Positron works equally well with R and Python, which is useful if you work across both
\end{itemize}

\textbf{Note}: All the command line instructions in this assignment work identically regardless of whether you use RStudio, Positron, or a standalone terminal. The differences only apply when using the graphical interface.

\clearpage
\section{Sublime Text 4}

Sublime Text is a fast, lightweight code editor. Unlike RStudio or Positron, it is not an IDE---it does not come with an R console, file browser, or plot viewer out of the box. Instead, you write R code in Sublime Text and send it to a separate R console running alongside it. This minimal setup requires only two packages.

\subsection{Step 1: Install Sublime Text 4}

Download from \url{https://www.sublimetext.com} and install it.

\subsection{Step 2: Install Package Control}

Package Control is Sublime Text's package manager. To install it:

\begin{enumerate}
  \item Open the Command Palette: \code{Ctrl+Shift+P} (Windows/Linux) or \code{Cmd+Shift+P} (Mac)
  \item Type ``Install Package Control'' and press Enter
  \item Wait for the confirmation message
\end{enumerate}

\subsection{Step 3: Install R-IDE and SendCode}

Using Package Control, install two packages:

\begin{enumerate}
  \item Open the Command Palette (\code{Ctrl+Shift+P} / \code{Cmd+Shift+P})
  \item Type ``Package Control: Install Package'' and press Enter
  \item Search for \textbf{R-IDE} and install it (provides R syntax highlighting, code completions, and function signatures)
  \item Repeat the process and install \textbf{SendCode} (sends code from the editor to an external R console)
\end{enumerate}

\subsection{Step 4: Configure SendCode}

SendCode needs to know where to send your code. Open its settings:

\begin{enumerate}
  \item Go to Preferences $\rightarrow$ Package Settings $\rightarrow$ SendCode $\rightarrow$ Settings
  \item In the right-hand pane (user settings), paste the following:
\end{enumerate}

\begin{lstlisting}
{
    "prog": "Terminal"
}
\end{lstlisting}

On \textbf{Mac}, set \code{"prog"} to \code{"Terminal"} to send code to the built-in Terminal (where you will run R), or to \code{"iTerm"} if you use iTerm2. On \textbf{Windows}, set it to \code{"Cmder"}, \code{"ConEmu"}, or another terminal emulator. On \textbf{Linux}, set it to \code{"tmux"} or \code{"linux-terminal"}.

\subsection{Step 5: The Workflow}

\begin{enumerate}
  \item Open a terminal window and start R by typing \code{R} and pressing Enter
  \item In Sublime Text, open your \code{.R} file
  \item Place your cursor on a line of code and press \code{Ctrl+Enter} (Windows/Linux) or \code{Cmd+Enter} (Mac)---SendCode sends that line to the R console and moves to the next line
  \item To send a selection, highlight multiple lines and press the same shortcut
\end{enumerate}

That is the entire setup. You edit in Sublime Text, execute in the R console, and switch between them as needed. For Git operations, use the terminal.

\end{document}