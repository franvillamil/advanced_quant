% Preamble for problem sets, exams, and project descriptions
% Applied Quantitative Methods for the Social Sciences II

\documentclass[12pt, a4paper]{article}
\usepackage[margin = 2cm]{geometry}
\usepackage{graphicx}
\usepackage[english]{babel}
\usepackage[utf8]{inputenc}
\usepackage[colorlinks = TRUE, linkcolor = blue, urlcolor = blue, citecolor = blue]{hyperref}
\usepackage{setspace}
\setstretch{1.25}
\renewcommand*\rmdefault{ppl}

\usepackage[]{titlesec}
    \titleformat*{\section}{\large\bf}
    \titleformat*{\subsection}{\normalsize\bf}
    \titleformat*{\subsubsection}{\normalsize\it}

% For code blocks
\usepackage{listings}
\usepackage{xcolor}
\lstset{
    basicstyle=\ttfamily\small,
    backgroundcolor=\color{gray!10},
    frame=single,
    framerule=0pt,
    breaklines=true,
    columns=fullflexible,
    keepspaces=true
}

% For inline code
\newcommand{\code}[1]{\texttt{#1}}

% Enumerate with letters
\usepackage{enumitem}

\textbf{\large Appendix to Assignment 1:\\\vspace{5pt}Setting up Github account}\\\vspace{10pt}
\end{center}

\tableofcontents

\section{Extra: Installing Git Locally}

If you want to use Git from the command line, you need to install it first.

\subsection{Installation}

\begin{itemize}
  \item \textbf{Mac}: Git comes pre-installed. Or install via Homebrew: \code{brew install git}
  \item \textbf{Windows}: Download from \url{https://git-scm.com/download/win}
  \item \textbf{Linux}: Use your package manager, e.g., \code{sudo apt install git}
\end{itemize}

\subsection{First-Time Configuration}

After installing, configure your identity (run once):

\begin{lstlisting}
git config --global user.name "Your Name"
git config --global user.email "your@email.com"
\end{lstlisting}

\subsection{Cloning an Existing Repository}

If you created the repository on GitHub first, download it to your computer:

\begin{lstlisting}
git clone https://github.com/YOUR-USERNAME/aqmss2.git
cd aqmss2
\end{lstlisting}

\subsection{Configuring RStudio}

To use Git in RStudio:
\begin{enumerate}
  \item Go to Tools $\rightarrow$ Global Options $\rightarrow$ Git/SVN
  \item Make sure ``Git executable'' points to your Git installation
  \item Restart RStudio
\end{enumerate}

\vspace{15pt}

\subsection*{Quick Reference on Common Git Commands}

\begin{tabular}{ll}
\textbf{Command} & \textbf{What it does} \\
\hline
\code{git status} & Show which files have changed \\
\code{git add <file>} & Stage a file for commit \\
\code{git add .} & Stage all changed files \\
\code{git commit -m "msg"} & Save staged changes with a message \\
\code{git push} & Upload commits to GitHub \\
\code{git pull} & Download changes from GitHub \\
\code{git log --oneline} & Show commit history (compact) \\
\code{git diff} & Show changes not yet staged \\
\end{tabular}

\clearpage
\section{Extra: Setting Up the Command Line on Windows}

If you are using Windows and want to follow the command line instructions in this assignment, you need to set up a Unix-like terminal first. The default Windows terminals (Command Prompt and PowerShell) use different commands and syntax from what is shown in this assignment and in most online tutorials.

\subsection{Recommended: Use Git Bash}

When you install Git for Windows (from \url{https://git-scm.com/download/win}), it includes \textbf{Git Bash}---a terminal emulator that provides a Unix-like command line environment on Windows. This is the easiest way to get started.

\begin{enumerate}
  \item Download and run the Git for Windows installer
  \item During installation, accept the default options. In particular:
    \begin{itemize}
      \item Select ``Use Git from Git Bash only'' (or the default option)
      \item Select ``Use bundled OpenSSH''
      \item Select ``Checkout Windows-style, commit Unix-style line endings''
    \end{itemize}
  \item After installation, you can open Git Bash by:
    \begin{itemize}
      \item Searching for ``Git Bash'' in the Start menu
      \item Right-clicking in any folder and selecting ``Git Bash Here''
    \end{itemize}
\end{enumerate}

Git Bash supports all the commands used in this assignment (\code{mkdir}, \code{cd}, \code{echo}, \code{git}, etc.) with the same syntax as on Mac and Linux.

\subsection{Alternative: Windows Terminal with PowerShell}

If you prefer to use PowerShell, be aware of these differences:

\begin{itemize}
  \item Most Git commands work the same (\code{git add}, \code{git commit}, etc.)
  \item File paths use backslashes (\code{\textbackslash}) instead of forward slashes (\code{/}), though Git accepts both
  \item The \code{echo} command syntax differs: use \code{Set-Content} or \code{Out-File} instead of \code{>} redirection
  \item Some Unix commands (\code{touch}, \code{cat}) are not available by default
\end{itemize}

\subsection{Setting up your PATH}

After installing Git, make sure it is accessible from your terminal:

\begin{lstlisting}
# In Git Bash or PowerShell, run:
git --version
\end{lstlisting}

If this returns a version number (e.g., \code{git version 2.43.0}), you are ready to go. If not, you may need to add Git to your system PATH:

\begin{enumerate}
  \item Open Settings $\rightarrow$ System $\rightarrow$ About $\rightarrow$ Advanced system settings
  \item Click ``Environment Variables''
  \item Under ``System variables'', find \code{Path} and click ``Edit''
  \item Add the path to your Git installation (usually \code{C:\textbackslash Program Files\textbackslash Git\textbackslash cmd})
  \item Click OK and restart your terminal
\end{enumerate}


\end{document}