\documentclass[11pt,a4paper]{article}

\usepackage[margin=2.5cm]{geometry}
\usepackage{amsmath}
\usepackage{amssymb}
\usepackage{hyperref}
\usepackage{enumitem}
\usepackage{booktabs}
\usepackage{graphicx}

\hypersetup{
    colorlinks=true,
    linkcolor=blue,
    urlcolor=blue,
    citecolor=blue
}

\title{\textbf{Problem Set 3: Applied Regression (II)}}
\author{Applied Quantitative Methods for the Social Sciences II}
\date{Due: February 26, 2026}

\begin{document}

\maketitle

\noindent\textit{Note: Please prepare your answers using R Markdown or Quarto and submit a PDF file. Include all code and R output used to answer the questions. For visualizations, please use the \texttt{marginaleffects} package.}

\section{Conceptual Questions}

\subsection{Question 1: Interaction Effects}

Consider the model: $Y = \beta_0 + \beta_1 X + \beta_2 Z + \beta_3 (X \times Z) + \varepsilon$

\begin{enumerate}[label=\alph*)]
    \item What is the marginal effect of $X$ on $Y$? Show how it depends on $Z$.
    \item Suppose $\beta_1 = 0.5$, $\beta_3 = -0.1$, and $Z$ ranges from 0 to 10. At what value of $Z$ does the effect of $X$ become zero?
    \item Why is it incorrect to interpret $\beta_1$ as ``the effect of $X$'' in this model?
    \item A researcher estimates this model and finds that $\beta_1$ is not statistically significant. They conclude that ``$X$ has no effect.'' Explain why this conclusion is problematic.
\end{enumerate}

\subsection{Question 2: Non-linear Relationships}

\begin{enumerate}[label=\alph*)]
    \item Consider the model $Y = \beta_0 + \beta_1 X + \beta_2 X^2 + \varepsilon$. What is the marginal effect of $X$ on $Y$? At what value of $X$ is the marginal effect zero?
    \item Suppose $\beta_1 = 2$ and $\beta_2 = -0.1$. Sketch the relationship between $X$ and $Y$. Is it U-shaped or inverted U-shaped?
    \item In the log-log model $\log(Y) = \beta_0 + \beta_1 \log(X) + \varepsilon$, what is the interpretation of $\beta_1$?
\end{enumerate}

\subsection{Question 3: Standard Errors and Inference}

\begin{enumerate}[label=\alph*)]
    \item Explain what heteroskedasticity is and why it affects standard errors but not point estimates.
    \item A researcher reports robust standard errors in their analysis. What does this mean? When should robust standard errors be used?
    \item Explain the difference between statistical significance and practical significance. Give an example where a result is statistically significant but not practically significant.
\end{enumerate}

\section{Applied Analysis: Gapminder Data}

For this problem set, you will use the Gapminder dataset, which contains country-level data on life expectancy, GDP per capita, and population over time.

\begin{verbatim}
install.packages("gapminder")
library(gapminder)
data(gapminder)
\end{verbatim}

You will also need the \texttt{marginaleffects} package for visualization:

\begin{verbatim}
install.packages("marginaleffects")
library(marginaleffects)
\end{verbatim}

\subsection{Question 4: Non-linear Relationships}

Focus on the year 2007 for this analysis.

\begin{enumerate}[label=\alph*)]
    \item Create a scatter plot of GDP per capita versus life expectancy. Based on the plot, do you think a linear model is appropriate?
    \item Estimate three models:
    \begin{itemize}
        \item Model 1: $\text{lifeExp} = \beta_0 + \beta_1 \cdot \text{gdpPercap} + \varepsilon$
        \item Model 2: $\text{lifeExp} = \beta_0 + \beta_1 \cdot \log(\text{gdpPercap}) + \varepsilon$
        \item Model 3: $\text{lifeExp} = \beta_0 + \beta_1 \cdot \text{gdpPercap} + \beta_2 \cdot \text{gdpPercap}^2 + \varepsilon$
    \end{itemize}
    Report the results in a single table.
    \item Compare the $R^2$ values of the three models. Which model fits the data best?
    \item Using Model 2, interpret the coefficient on $\log(\text{gdpPercap})$. What happens to life expectancy when GDP per capita doubles?
\end{enumerate}

\subsection{Question 5: Interaction Effects}

Now we will examine whether the relationship between GDP and life expectancy varies by continent.

\begin{enumerate}[label=\alph*)]
    \item Estimate a model with $\log(\text{gdpPercap})$, continent, and their interaction:
    $$\text{lifeExp} = \beta_0 + \beta_1 \cdot \log(\text{gdpPercap}) + \beta_2 \cdot \text{continent} + \beta_3 \cdot \log(\text{gdpPercap}) \times \text{continent} + \varepsilon$$
    Report the results.
    \item Using the \texttt{marginaleffects} package, calculate the marginal effect of $\log(\text{gdpPercap})$ for each continent. Report the estimates and 95\% confidence intervals.
    \item Create a plot showing the predicted life expectancy across the range of GDP per capita for each continent. Include confidence bands.
    \item Interpret your findings substantively. Does the relationship between wealth and health differ by continent? What might explain these differences?
\end{enumerate}

\subsection{Question 6: Presenting Results}

\begin{enumerate}[label=\alph*)]
    \item Using your interaction model from Question 5, calculate the predicted life expectancy for:
    \begin{itemize}
        \item A poor African country (GDP per capita = \$1,000)
        \item A middle-income Asian country (GDP per capita = \$10,000)
        \item A wealthy European country (GDP per capita = \$40,000)
    \end{itemize}
    Report both point estimates and 95\% confidence intervals.
    \item Calculate the ``first difference'': How much higher is life expectancy in a wealthy European country compared to a poor African country? Report the 95\% confidence interval for this difference.
    \item Create a visualization that effectively communicates the key findings from your analysis. Explain why you chose this particular visualization.
\end{enumerate}

\subsection{Question 7: Diagnostics}

\begin{enumerate}[label=\alph*)]
    \item Create a residuals vs. fitted values plot for your interaction model. Are there any patterns that suggest model misspecification?
    \item Test for heteroskedasticity using the Breusch-Pagan test. Report the test statistic and p-value.
    \item Re-estimate your model using robust standard errors. How do the confidence intervals change?
\end{enumerate}

\section{Synthesis Question}

\subsection{Question 8}

In approximately 300 words, discuss the importance of moving beyond simple regression tables when communicating research findings. Drawing on your analysis of the Gapminder data, explain:

\begin{itemize}
    \item Why predicted values and marginal effects are more informative than regression coefficients alone
    \item The value of visualizations in understanding complex relationships (like interactions)
    \item How to effectively communicate uncertainty in your estimates
\end{itemize}

\vspace{1cm}

\noindent\textbf{Submission:} Submit your PDF file through Aula Global by the start of class on February 26, 2026.

\end{document}
