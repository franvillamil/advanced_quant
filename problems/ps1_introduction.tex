% Preamble for problem sets, exams, and project descriptions
% Applied Quantitative Methods for the Social Sciences II

\documentclass[12pt, a4paper]{article}
\usepackage[margin = 2cm]{geometry}
\usepackage{graphicx}
\usepackage[english]{babel}
\usepackage[utf8]{inputenc}
\usepackage[colorlinks = TRUE, linkcolor = blue, urlcolor = blue, citecolor = blue]{hyperref}
\usepackage{setspace}
\setstretch{1.25}
\renewcommand*\rmdefault{ppl}

\usepackage[]{titlesec}
    \titleformat*{\section}{\large\bf}
    \titleformat*{\subsection}{\normalsize\bf}
    \titleformat*{\subsubsection}{\normalsize\it}

% For code blocks
\usepackage{listings}
\usepackage{xcolor}
\lstset{
    basicstyle=\ttfamily\small,
    backgroundcolor=\color{gray!10},
    frame=single,
    framerule=0pt,
    breaklines=true,
    columns=fullflexible,
    keepspaces=true
}

% For inline code
\newcommand{\code}[1]{\texttt{#1}}

% Enumerate with letters
\usepackage{enumitem}


\begin{document}

\thispagestyle{empty}

\begin{center}
\textbf{\Large Problem Set 1: Introduction to Git and GitHub}\\\vspace{10pt}
{\Large Applied Quantitative Methods for the Social Sciences II}\\\vspace{10pt}
Carlos III--Juan March Institute, Spring 2026
\end{center}

\vspace{10pt}
\noindent
\textbf{\large Instructions:}

\vspace{10pt}
\begin{itemize}
\setlength\itemsep{0pt}
  \item {\color{red}{\textbf{Deadline}}}: \textbf{February 12, before class}
  \item This problem set introduces you to Git and GitHub, which you will use throughout the course to submit your work
  \item For this course, you will submit all assignments as plain \code{.R} files in a public GitHub repository
  \item Complete all the tasks below and send me the link to your GitHub repository
\end{itemize}

\vspace{20pt}

\section{What is Version Control?}

Version control is a system that records changes to files over time. It allows you to:

\begin{itemize}
  \item Track the history of your project
  \item Revert to previous versions if something goes wrong
  \item Collaborate with others without overwriting each other's work
  \item Keep a backup of your work in the cloud
\end{itemize}

\textbf{Git} is the most widely used version control system. It tracks changes locally on your computer.

\textbf{GitHub} is a web platform that hosts Git repositories online, making it easy to share code, collaborate, and back up your work.

\vspace{10pt}

\section{Why Use Git and GitHub?}

In research and data analysis, version control is essential for:

\begin{itemize}
  \item \textbf{Reproducibility}: Others (and your future self) can see exactly what you did
  \item \textbf{Collaboration}: Multiple people can work on the same project
  \item \textbf{Backup}: Your work is safely stored online
  \item \textbf{Transparency}: Open science practices require sharing code and data
\end{itemize}

Many journals and research groups now require or encourage sharing code via GitHub.

\vspace{10pt}

\section{The Basic Git Workflow}

Git works with three main concepts:

\begin{enumerate}
  \item \textbf{Working directory}: The files on your computer
  \item \textbf{Staging area}: Files you've marked to be included in the next snapshot
  \item \textbf{Repository}: The history of all snapshots (commits)
\end{enumerate}

The basic workflow is:

\begin{enumerate}
  \item Make changes to your files
  \item \textbf{Stage} the changes you want to save: \code{git add filename}
  \item \textbf{Commit} the staged changes with a message: \code{git commit -m "description"}
  \item \textbf{Push} your commits to GitHub: \code{git push}
\end{enumerate}

\vspace{10pt}

\section{Ways to Use Git}

There are several ways to interact with Git and GitHub. All of them do the same thing---choose whichever you find most comfortable, or mix and match.

\subsection{Option 1: GitHub Web Interface}

The simplest way to get started. You can create repositories, upload files, and make commits directly on \url{https://github.com}. Good for beginners, but limited for complex workflows.

\subsection{Option 2: Command Line (Terminal)}

The most powerful and flexible option. You use commands like \code{git add}, \code{git commit}, and \code{git push} in your terminal (Terminal on Mac, Git Bash on Windows). This is what most experienced users prefer.

\subsection{Option 3: RStudio Integration}

If you use RStudio, it has built-in Git support. You can stage, commit, and push using buttons in the Git pane (top-right by default). This is convenient if you're already working in RStudio. To enable it, you need to have Git installed and configure RStudio to recognize it (Tools $\rightarrow$ Global Options $\rightarrow$ Git/SVN).

\vspace{15pt}

\section{Tasks}

Complete the following tasks. You can use any of the methods described above (web, command line, or RStudio).

\vspace{10pt}

\subsection{Task 1: Create a GitHub Account}

If you don't already have one, create a free account at \url{https://github.com}.

\begin{itemize}
  \item Choose a professional username (you may use this for years)
  \item Add a profile picture and brief bio if you like
\end{itemize}

\vspace{10pt}

\subsection{Task 2: Create a Repository}

Create a new \textbf{public} repository for this course. Call it something like \code{aqmss2} or \code{quant-methods-2026}.

\textbf{Via the web interface:}
\begin{enumerate}
  \item Click the ``+'' icon in the top right, then ``New repository''
  \item Enter a name and description
  \item Make sure ``Public'' is selected
  \item Check ``Add a README file''
  \item Click ``Create repository''
\end{enumerate}

\textbf{Via the command line:}
\begin{lstlisting}
mkdir aqmss2
cd aqmss2
git init
echo "# AQMSS II" > README.md
git add README.md
git commit -m "Initial commit"
git branch -M main
git remote add origin https://github.com/YOUR-USERNAME/aqmss2.git
git push -u origin main
\end{lstlisting}

\vspace{10pt}

\subsection{Task 3: Edit the README}

The README file is the ``front page'' of your repository. Edit it to include:

\begin{itemize}
  \item Your name
  \item A brief description of the repository (e.g., ``Problem sets for AQMSS II, Spring 2026'')
  \item Optionally, a table of contents or list of files
\end{itemize}

\textbf{Via the web:} Click on \code{README.md}, then the pencil icon to edit. When done, scroll down and click ``Commit changes.''

\textbf{Via command line:} Edit the file locally, then:
\begin{lstlisting}
git add README.md
git commit -m "Update README with course info"
git push
\end{lstlisting}

\vspace{10pt}

\subsection{Task 4: Create a Folder for Problem Sets}

Create a folder called \code{problem\_sets} (or similar) in your repository. Inside it, create a file called \code{ps1.R} with a comment header:

\begin{lstlisting}
# Problem Set 1
# AQMSS II, Spring 2026
# [Your Name]

# This file will contain my solutions for PS1
\end{lstlisting}

Commit this file to your repository.

\vspace{10pt}

\subsection{Task 5: Explore the History}

Look at the commit history of your repository:

\textbf{Via the web:} Click on ``Commits'' (or the clock icon) to see all your commits.

\textbf{Via command line:}
\begin{lstlisting}
git log --oneline
\end{lstlisting}

Take a screenshot of your commit history showing at least 2--3 commits.

\vspace{15pt}

\section{Submission}

Send me an email with:

\begin{enumerate}
  \item The URL of your GitHub repository (e.g., \code{https://github.com/username/aqmss2})
  \item The screenshot of your commit history
\end{enumerate}

I will check that your repository is public and contains a README and at least one \code{.R} file.

\vspace{15pt}

\section{Optional: Setting Up Git Locally}

If you want to use Git from the command line or RStudio, you'll need to set it up on your computer:

\begin{enumerate}
  \item \textbf{Install Git}: Download from \url{https://git-scm.com/downloads}
  \item \textbf{Configure your identity}:
\begin{lstlisting}
git config --global user.name "Your Name"
git config --global user.email "your@email.com"
\end{lstlisting}
  \item \textbf{Clone your repository} (download it to your computer):
\begin{lstlisting}
git clone https://github.com/YOUR-USERNAME/aqmss2.git
\end{lstlisting}
\end{enumerate}

\textbf{For RStudio users}: Go to File $\rightarrow$ New Project $\rightarrow$ Version Control $\rightarrow$ Git, and paste your repository URL. RStudio will clone the repository and set up a project for you.

\vspace{15pt}

\section{Resources}

\begin{itemize}
  \item GitHub's official guides: \url{https://docs.github.com/en/get-started}
  \item Happy Git with R (excellent for R users): \url{https://happygitwithr.com}
  \item Git cheat sheet: \url{https://education.github.com/git-cheat-sheet-education.pdf}
  \item Pro Git book (free online): \url{https://git-scm.com/book/en/v2}
\end{itemize}

\vspace{10pt}
\noindent
\textit{We will cover more advanced Git workflows and reproducible computing practices later in the course (Session 5). For now, just get comfortable with the basics!}

\end{document}
